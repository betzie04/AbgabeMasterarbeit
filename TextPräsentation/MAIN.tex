% Die Beamer-Klasse unterstützt folgende Optionen, die von
% besonderem Interesse sind (alle Standardoptionen werden
% ebenfalls unterstützt; siehe beamer-Basisdokumentation):
% 
%%%%%%%%%%%%%%%%%%%%%%%%%%%%%%%%%%%%%%%%%%%%%%%%%%%%%%%%%%%%%%%
% aspectratio: Seitenverhältnis des resultierenden Dokuments
% (Achtung: Aufgrund der Designvorgaben ergeben sich unterschiedliche
% Größen der effektiv nutzbaren Textblöcke)
%
% Standardeinstellung: 'aspectratio=43'
%
% Mögliche Einstellungen:
% 'aspectratio=43'   (4:3)
% 'aspectratio=169'  (16:9)
% 'aspectratio=1610' (16:10)
% 
%%%%%%%%%%%%%%%%%%%%%%%%%%%%%%%%%%%%%%%%%%%%%%%%%%%%%%%%%%%%%%%
% fontsize: Basisschriftgröße (Größen für Überschriften etc. werden
% aus dieser Basis automatisch abgeleitet)
%
% Standardeinstellung: '22pt' (entspricht den Design-Vorgaben; sehr groß!)
%
% Mögliche Einstellungen:
% '8pt', '9pt', '10pt', '11pt', '12pt', '14pt', '16pt',
% '17pt','20pt','22pt', '24pt', '26pt', '28pt'


\documentclass[aspectratio=169,16pt]{beamer}

% Der OTHR-Theme unterstützt folgende Optionen:
% 
%%%%%%%%%%%%%%%%%%%%%%%%%%%%%%%%%%%%%%%%%%%%%%%%%%%%%%%%%%%%%%%
% department: (Wahl der Abteilung/Fakultät)
%
% default: 'OTHR'
%
% Mögliche Einstellungen:
% 'FakA', 'FakAM', 'FakB', 'FakBW', 'FakEI', 
% 'FakIM', 'FakM', 'FakS', 'ZWW', 'IPF',
% 'SappZ', 'KNB', 'ReMIC', 'LFD', 'LAS3',
% 'DK0PT', 'LBM', 'LeanLab', 'LFT', 'LFW',
% 'LMP', 'LMS', 'LRT', 'LWS', 'RRRU',
% 'RST', 'CEEC', 'FEM', 'IST'
%%%%%%%%%%%%%%%%%%%%%%%%%%%%%%%%%%%%%%%%%%%%%%%%%%%%%%%%%%%%%%%%%
% headerMode: Aussehen und Inhalt der Kopfleiste
%
% Standardeinstellung: 'full'
%
% Mögliche Einstellungen:
% 'full', 'frametitle', 'frametitleSection'
% 

%%%%%%%%%%%%%%%%%%%%%%%%%%%%%%%%%%%%%%%%%%%%%%%%%%%%%%%%%%%%%%%%%
% Binäre Schalter (können angegeben oder nicht angegeben werden;
% Standardeinstellung: Nicht angegeben)
%
%%%%%%%%%%%%%%%
% navbar: Navigationssymbole anzeigen (Seite vor/zurück, Kapitel vor/zurück etc.)

% pageNumbers: Seitennummerierung

% blackFont: Nur schwarze Schriftfarbe verwenden (ansonsten: Fakultätsfarben)

% frametitleCenter: Titel in der Kopfzeile zentrieren (ansonsten: rechtsbündig)

\usetheme[department=FakIM,pageNumbers]{OTHR}

% Inhaltsspezifische Zusatzpakete laden 
\usepackage[ngerman]{babel}
\usepackage[utf8]{luainputenc}
\usepackage[ngerman]{babel}
\usepackage[utf8]{luainputenc}
\usepackage{filecontents}
\usepackage{subfigure}
\usepackage{numprint}
\usepackage{tikz}
\usepackage{tikzpagenodes}
\usetikzlibrary{calc, shapes, backgrounds, arrows}
\usepackage{ragged2e} 
\usepackage{blindtext}
\usepackage{enumerate}
\usepackage{calc}
\usepackage{graphicx}
\usepackage[absolute,overlay]{textpos}  % in der Präambel
\usepackage{tikz}
\usepackage{algorithm}
\usepackage{algpseudocode}
\usepackage{amsmath}  
\usepackage{listings}
\usetikzlibrary{positioning}
\usepackage{booktabs}
\usepackage{multirow}

\usepackage[most]{tcolorbox}


\begin{document}


\lstdefinestyle{mystyle}{
	language=Python,
	basicstyle=\ttfamily\small,
	keywordstyle=\color{blue},
	commentstyle=\color{gray},
	stringstyle=\color{orange},
	showstringspaces=false,
	breaklines=true,
	frame=singe
}
\renewcommand\lstlistingname{Code}

\lstset{style=mystyle}

\title{Präsentation der Masterarbeit}
\subtitle{On Homotopy of Similar Neural Networks using Non-linear Transformations}
\author{Bettina Zieger}

\begin{textblock*}{5cm}(12cm,0cm)  % Breite, (x,y)-Position in cm
	\includegraphics[width=0.7\linewidth]{BilderPräsentation/aisec_logo.pdf}
\end{textblock*}
	


\institute{Fakultät Informatik und Mathematik}
\date{24.07.2025}

\begin{filecontents}{\jobname.bib}
@inproceedings{
chan_affine_2024,
title={On Affine Homotopy between Language Encoders},
author={Robin Chan and Reda Boumasmoud and Anej Svete and Yuxin Ren and Qipeng Guo and Zhijing Jin and Shauli Ravfogel and Mrinmaya Sachan and Bernhard Sch{\"o}lkopf and Mennatallah El-Assady and Ryan Cotterell},
booktitle={The Thirty-eighth Annual Conference on Neural Information Processing Systems},
year={2024},
url={https://openreview.net/forum?id=FTpOwIaWUz}
}

\end{filecontents}

% Zitationsstil wählen: amsalpha
\bibliographystyle{plain}

\maketitle
\frame{\tableofcontents \begin{textblock*}{5cm}(12cm,0cm)  % Breite, (x,y)-Position in cm
		\includegraphics[width=0.7\linewidth]{BilderPräsentation/aisec_logo.pdf}
\end{textblock*}}

\section{Motivation}
\frame{\tableofcontents[currentsection]}
% --- Folie 2: Forschungsfrage ---
\begingroup
\frametitle{Fraunhofer AISEC}
\begin{frame}
	\begin{figure}
		\centering
		\includegraphics[width=0.7\linewidth]{BilderPräsentation/AISEC.png}
		\caption{Fraunhofer-Institut in Garching}
		\label{fig:enter-label}
	\end{figure}
	\vspace{0.2em}
	\begin{itemize}
		\item Fraunhofer-Institut für Angewandte und Integrierte Sicherheit in Garching
		\item Cognitive Security Technologies
		\begin{itemize}
			\item Künstliche Intelligenz und IT-Sicherheit
		\end{itemize}
	\end{itemize}
\end{frame}
\endgroup


\begingroup
\frametitle{Motivation}
\begin{frame}
	\begin{itemize}
		
		\uncover<1->{
			\item[] Neuronale Netze dominieren viele Bereiche:
			\item[]
			{\scriptsize
				\begin{tabular}{@{}p{0.45\textwidth}@{\hspace{1.4em}}p{0.45\textwidth}@{}}
					\hspace{1.4em}\includegraphics[height=1em]{BilderPräsentation/image.png} Bildverarbeitung &
					\includegraphics[height=1em]{BilderPräsentation/alert.png} Anomalieerkennung \\
					\hspace{1.4em}\includegraphics[height=1em]{BilderPräsentation/text.png} Textverarbeitung &
					\includegraphics[height=1em]{BilderPräsentation/predict.png} Klassifikation und Vorhersage
				\end{tabular}
		}}
		
		\vspace{0.4em}
		
		\uncover<1->{
			\item[] \includegraphics[height=1em]{BilderPräsentation/network.png} Sie lernen komplexe, nichtlineare Funktionen~\cite{antiga_deep_2020}}
		
		\vspace{0.4em}
		
		\uncover<2->{
			\item[] \includegraphics[height=1em]{BilderPräsentation/question.png} Wie ähnlich sind zwei Netze?
			\begin{itemize}
				\item Verständnis interner Repräsentationen
				\item Gleiche Architektur
				\item Unterschiedliches Training
		\end{itemize}}
		
	
		
	\end{itemize}
	{\tiny Quelle: Antiga et al., “Deep Learning with PyTorch”, Manning 2020~\cite{antiga_deep_2020}}
\end{frame}
\endgroup

\begingroup
\frametitle{Forschungsansatz}
\begin{frame}
\begin{itemize}
    \item \textbf{Ausgangspunkt}: Chan et al.~\cite{chan_affine_2024} schlagen einen Homotopie-basierten Ansatz zur Modellvergleichbarkeit vor:
    \begin{itemize}
      \item \textbf{Intrinsic Homotopy:} Wie gut kann eine Repräsentation in eine andere transformiert werden?
      \item \textbf{Extrinsic Homotopy:} Wie ähnlich ist das Modellverhalten bei gleichem Downstream-Task?
    \end{itemize}
    
    \vspace{0.8em}
    \item \textbf{Limitation}: Nur \emph{affine} Transformationen $\Rightarrow$ begrenzte Ausdruckskraft für komplexe Modelle.
    \end{itemize}
\vfill
{\tiny Quelle: Chan et al., “On Affine Homotopy between Language Encoders”, NeurIPS 2024~\cite{chan_affine_2024}}
\end{frame}
\endgroup

\begingroup
\frametitle{Forschungsfragen}
\begin{frame}
\vspace{0.8em}
\textbf{Forschungsfragen:}
\begin{itemize}
  \item Wie lassen sich intrinsic und extrinsic Homotopy im nichtlinearen Fall definieren?
  \item Erfassen nichtlineare Transformationen Ähnlichkeit besser als lineare?
  \item Besteht ein konsistenter Zusammenhang zwischen intrinsic und extrinsic Homotopy?
\end{itemize}
\end{frame}
\endgroup


\section{Neuronale Netze}
\frame{\tableofcontents[currentsection]}
\section{Transformer Models}\label{TM}
Transformer models have become the foundation of modern \ac{NLP} due to their ability to capture complex relationships between words — even when those words are far apart in a sentence.
Unlike older models such as \ac{RNN}, which process text sequentially and struggle with long-range dependencies, Transformers process entire sequences in parallel.
Their core innovation is the \textbf{self-attention} mechanism, which allows the model to weigh the relevance of each word with respect to others in the sequence.

A Transformer is a deep neural network architecture introduced by Vaswani et al.~\cite{vaswani_attention_2023}.
It has since become the basis of widely-used models such as BERT and GPT, which are trained on large corpora in a self-supervised manner, that is, without labeled data.

Transformers are primarily composed of an encoder and a decoder block, as illustrated in Figure~\ref{fig:encoderDecoder}.
The encoder processes the input sequence and generates contextual representations.
The decoder then uses these representations, along with previously generated tokens, to produce the output sequence.
This architecture enables applications such as machine translation, summarization, and text generation.


%This indicates that the model is optimize to develop an understanding of its inputs. 
%The decoder creates a target sequence using the encoder's representation (features) and other inputs. 
%This indicates that the model is optimized to generate an output. 

\begin{figure}[htb]
    \centering
    \includegraphics[width=0.45\linewidth]{Abschlussarbeit/Pictures/encoder_decoder_simple (1).pdf}
    \caption{Encoder-decoder structure of a Transformer.}
    \label{fig:encoderDecoder}
\end{figure}

A key strength of Transformer models is their self-attention mechanism, which enables them to weigh different parts of the input when making predictions.

For example, in the sentence:
\textit{``The cat chased the mouse because she was hungry.''
}
a Transformer can learn that ``she`` refers to ``the cat`` by attending to all words in the sequence at once.

%There are different models that are based on transformer. 

%\textbf{{BERT}} \cite{BERT} is an encoder model and uses only the encoder of a Transformer model. 
%BERT stands for \textbf{B}idirectional \textbf{E}ncoder \textbf{R}epresentations from \textbf{T}ransformers.
%BERT is designed to pre-train deep bidirectional representations from unlabeled text.
%Unlike recent language representation models \cite{peters-etal-2018-deep, GPT}, BERTS attention layer can access all words in the initial sentence in each step.
%Therefore, the pre-trained BERT model can be optimized without significant task-specific architectural changes with just one more output layer to create state-of-the-art models for a variety of tasks, including language inference and question answering.
%Models like that are often characterized as having a bi-directional attention and are called auto-encoding models.
%For tasks that require an understanding of the whole sentence, such as sentence classification, named entity recognition and extractive question answering are encoder models best suited.

%\textbf{GPT} \cite{GPT} is a decoder model and only uses the decoder of a Transformer model and a shortcut for Generative Pre-Training.
%GPT is a semi-supervised approach to language understanding tasks and is trained through a combination of unsupervised pre-training and supervised fine-tuning. 
%The attention layer can access only the words positioned before a given word in the sentence and are often called auto-regressive models.
%The goal is to learn a universal representation that can be easily adapted to different tasks. The approach involves a two-stage training procedure in which the initial parameters of the neural network model are learned on unlabeled data and adapted to the target tasks.


Two of the most influential Transformer-based language models are \textbf{BERT} and \textbf{GPT}, which differ in their architecture and training objective.

\paragraph{BERT – Bidirectional Encoder Representations from Transformers}
BERT~\cite{BERT} is an \textit{encoder-only} model. It processes the entire input sentence at once using bidirectional self-attention and is designed to learn contextual representations from unlabeled text through a masked language modeling objective.

Unlike earlier models~\cite{peters-etal-2018-deep, GPT}, BERT’s attention layers can access all tokens in the sequence during training. This allows it to capture both left and right context simultaneously. Models of this type are referred to as \textit{auto-encoding models}.
Thanks to its pre-training on large corpora and its architecture, BERT can be fine-tuned with minimal task-specific modifications for a wide range of NLP tasks, including sentence classification, named entity recognition, and extractive question answering.

\paragraph{GPT – Generative Pre-Training}
GPT~\cite{GPT} is a \textit{decoder-only} model. It uses unidirectional (left-to-right) attention and is trained to predict the next word in a sentence, making it well suited for text generation tasks.

GPT follows a two-stage training approach:
\begin{itemize}
    \item \textbf{Unsupervised pre-training} on raw text to learn general-purpose language representations.
    \item \textbf{Supervised fine-tuning} on task-specific labeled data.
\end{itemize}

Because GPT can only attend to preceding tokens at each position, it is called an \textit{auto-regressive model}.

%\paragraph{Summary of Model Types}
%\begin{itemize}
%    \item \textbf{BERT} (encoder-only) $\rightarrow$ for \textit{understanding} text.
%    \item \textbf{GPT} (decoder-only) $\rightarrow$ for \textit{generating} text.
%\end{itemize}

%\vspace{0.5em}
%\noindent


\section{Ähnlichkeitsanalyse Neuronaler Netze}
\frame{\tableofcontents[currentsection]}

\begingroup
\frametitle{ Ziel der Similarity Analysis}
% Similarity Analysis
\begin{frame}
	
	\begin{minipage}[t]{0.48\linewidth}
		\vspace*{-0.5em}
  \begin{itemize}
	\item Functional Similarity:
	\begin{itemize}
		\item Vergleich der Modellverhalten bei gleichen Aufgaben
		\item Sind die Ausgaben \( f(x) \) und \( f'(x) \) ähnlich?
	\end{itemize}
	\vspace{1em}
	\item Representational Similarity:
	\begin{itemize}
		\item Vergleich der internen Repräsentationen (z.B. Hidden States)
		\item Wie ähnlich sind die Zwischenrepräsentationen der Layer?
	\end{itemize}
\end{itemize}
	\end{minipage}
	\hfill
	\begin{minipage}[t]{0.49\linewidth}
		\vspace*{-0.5em}
		\includegraphics[width=1.2\linewidth]{BilderPräsentation/Similarity90Deg.drawio.pdf}
	\end{minipage}
	
\end{frame}
\endgroup


\begingroup
\frametitle{Vergleich durch internen Repräsentationen}
% Similarity Analysis
\begin{frame}

\uncover<1->{\begin{itemize}
	\item Repräsentation eines Modells $f$ auf Layer $l$
	\[
	R := R^{(l)} = (f^{(l)} \circ \dots \circ f^{(1)})(X) \in \mathbb{R}^{N \times D}
	\]
	\item $N$: Anzahl der Eingaben, $D$: Anzahl der Neuronen im Layer
	\item Repräsentationen \( R \) und \( R' \) können semantisch äquivalent sein, auch wenn sie sich numerisch unterscheiden.
	
\end{itemize}}

\vspace{-0.1em}
\uncover<2->{\begin{block}{Representational Similarity Measure}
		Eine Abbildung \( m(R, R') \) bewertet, wie ähnlich zwei Repräsentationen \( R, R' \in \mathbb{R}^{N \times D} \) sind.\\[0.2em]
		
\end{block}}

\vfill
{\tiny Quelle: Klabunde et al., “Similarity of Neural Network Models: A Survey of Functional and Representational Measures”, ACM Comput. Surv. 2020~\cite{klabunde_similarity_2024}}
\end{frame}
\endgroup

%\begingroup
%\frametitle{Vergleich durch internen Repräsentationen}
% Similarity Analysis
%\begin{frame}
%	\vspace{0.8em}
%\begin{block}{Invarianz eines Maßes}
%	Eine Abbildung $m$ ist \textbf{invariant} gegenüber $\mathcal{T}$, wenn \[
%	m(R, R') = m(\varphi(R), \varphi'(R')) \quad \forall \varphi, \varphi' \in \mathcal{T}\]
%\end{block}

%\vspace{0.8em}
%\begin{itemize}
%	\item Zwei Repräsentationen gelten als \textbf{$\mathcal{T}$-äquivalent}, wenn $\exists \psi \in \mathcal{T}$ mit $\psi(R) = R'$.
%	\item Schwächung des Metrik-Axioms:

%	\[
%	m(R, R') = 0 \iff R \sim_\mathcal{T} R'
%	\]
	%\textit{(Identität nur bis auf Transformationen)}
 %\end{itemize}
%\end{frame}
%\endgroup


\section{Intrinsic Homotopy}
\frame{\tableofcontents[currentsection]}
To assess the effectiveness of our extended framework, we begin by replicating the intrinsic homotopy experiment from Chan et al.~\cite{chang_clust_met_space}.
Our goal is to directly compare the representational alignment achieved by affine and non-linear transformations.

We aimed to reproduce the results reported in the original work as closely as possible.
In a first attempt, we applied the architecture and loss computation as described in the original paper, and used the train data for training and test data for evaluation. 
However, this setting did not yield meaningful approximation of the target representations.
In a second attempt, we trained the transformations on the training set, again using the original architecture and per-sample maximum loss. This setup led to a close reproduction of the results reported in the paper: for most tasks, we observed very similar approximation distances on the training set. The only deviations were slightly lower distances for \texttt{SST-2} and slightly higher values for \texttt{RTE}, as measured by the median.
Figure~\ref{fig:baseline-reproduction-trainings-loss} confirms that our reimplementation closely matches the reported training performance

\begin{figure}[h]
	\centering
	\includegraphics[width=\linewidth]{Abschlussarbeit/Pictures/New_all_in_one_scatter_lower_height.png}
	\caption{Intrinsic approximation distances on the training set for affine and non-linear transformations under the original training scheme.}
	\label{fig:baseline-reproduction-trainings-loss}
\end{figure}

To ensure a rigorous and transparent evaluation protocol for the subsequent experiments, we now consistently train on the training split, use early stopping based on validation loss, and report final results on a held-out test set.

After verifying that our implementation reproduces the reported approximation behavior on the training set, we next examine the intrinsic homotopy distances across a range of learning rates for both affine and non-linear transformations.

In Section~\ref{IH}, we formalized the concept of intrinsic homotopy using linear mappings \( \psi \in S \subset \text{Aff}(V) \), and in Section~\ref{InrinsicHomotopyonNonlinTransf} extended this framework to the nonlinear case \( \psi \in  \mathcal{F}_{\text{Lip}_1} \subset \mathcal{C}_{V,W} \), where \(  \mathcal{F}_{\text{Lip}_1} \subset \mathcal{C} \) consists of 1-Lipschitz continuous neural networks.

To empirically compare the representational alignment induced by these transformation classes, we compute the intrinsic homotopy distances \( d_{\mathrm{Aff}} := d_{\mathcal{\text{Aff}}}(h, g) \) and \( d_{\mathcal{F}_{\mathrm{Lip}_1}} := d_{\mathcal{F}_{\mathrm{Lip}_1}}(h, g) \) for each pair of encoders \( h, g \in \mathcal{E}_V \).
The mean of all distances is visualized for each learningrate and task in Figure~\ref{fig:lin_vs_nonlind_dist}.

\begin{figure}[h]
    \centering
    \includegraphics[width=\linewidth]{Abschlussarbeit/Pictures/intrinsic_distance_all_tasks.png}
    \caption{Intrinsic Distances across tasks and learning rates for linear and non-linear models}
    \label{fig:lin_vs_nonlind_dist}
\end{figure}

In addition to the homotopy distances, we report Spearman’s $\rho$ and Pearson’s $r$ correlations between predicted and target representations for both transformation regimes in Table~\ref{tab:correlation-transposed}.  
These metrics serve as complementary indicators of representational alignment: while the homotopy distance measures global approximation quality, the correlation coefficients quantify local relational consistency between vector pairs.

Spearman’s $\rho$ captures the rank-order correlation between the transformed representation $\psi(g)$ and the target $h$. 
As a rank-based measure, it is robust to monotonic non-linear distortions and reflects whether the relative structure between examples is preserved.

In contrast, Pearson’s $r$ quantifies the linear correlation between vectors.  
To assess whether a meaningful linear mapping between model representations exists, we compute $r(g, h)$, the Pearson correlation between the hidden states of the source and target models.  
Consistently low values of $r(g, h)$ across all tasks indicate that no strong global linear relationship exists between $g$ and $h$, implying that affine transformations are fundamentally limited in their ability to approximate one encoder from another.


While we include Pearson’s $r$ for interpretability, we do not use it as a basis for comparing model performance across tasks.

\begin{table}[h]
\centering
\caption{Correlation scores \(\rho, r\) across tasks and learning rates for linear and non-linear models.}
\label{tab:correlation-transposed}
\begin{tabular}{l c ccccccc}
\toprule
&\textbf{lr} & \texttt{QNLI} & \texttt{CoLA} & \texttt{MRPC} & \texttt{QQP} & \texttt{RTE} & \texttt{SST-2} \\
\midrule
\multirow{5}{*}{\(\rho_{\mathrm{Aff}}\)} 
& \(1\cdot10^{-5}\) & 0.44 & 0.29 & 0.47 & 0.44 & 0.44 & 0.32 \\
& \(1\cdot10^{-4}\) & 0.48 & 0.33 & 0.51 & 0.48 & 0.47 & 0.35 \\
& \(1\cdot10^{-3}\) & 0.47 & 0.33 & 0.51 & 0.47 & 0.47 & 0.35 \\
& \(1\cdot10^{-2}\) & 0.33 & 0.23 & 0.36 & 0.34 & 0.32 & 0.23 \\
& \(1\cdot10^{-1}\) & 0.089 & 0.058 & 0.089 & 0.094 & 0.080 & 0.054 \\
\midrule
\multirow{5}{*}{\(\rho_{\mathcal{C}}\)} 
& \(1\cdot10^{-5}\) & 0.089 & 0.11 & 0.12 & 0.11 & 0.11 & 0.10 \\
& \(1\cdot10^{-4}\) & 0.079 & 0.13 & 0.18 & 0.10 & 0.12 & 0.11 \\
& \(1\cdot10^{-3}\) & 0.026 & 0.079 & 0.0054 & 0.066 & 0.021 & 0.071 \\
& \(1\cdot10^{-2}\) & 0.014 & 0.061 & 0.0037 & 0.043 & 0.012 & 0.049 \\
& \(1\cdot10^{-1}\) & –0.008 & –0.018 & –0.0046 & –0.018 & –0.0051 & –0.023 \\
\midrule
\(r\) 
& &  \(5\cdot10^{-6}\) & \(4\cdot10^{-6}\) & \(4\cdot10^{-6}\) & \(5\cdot10^{-6}\) & \(3\cdot10^{-6}\) & \(5\cdot10^{-6}\)\\
%\multirow{5}{*}{\(r\)} 
%& \(1\cdot10^{-5}\) & \(5.2\cdot10^{-6}\) & \(4.4\cdot10^{-6}\) & \(3.6\cdot10^{-6}\) & \(4.5\cdot10^{-6}\) & \(2.5\cdot10^{-6}\) & \(4.8\cdot10^{-6}\) \\
%& \(1\cdot10^{-4}\) & \(5.2\cdot10^{-6}\) & \(4.4\cdot10^{-6}\) & \(3.6\cdot10^{-6}\) & \(4.5\cdot10^{-6}\) & \(2.5\cdot10^{-6}\) & \(4.8\cdot10^{-6}\) \\
%& \(1\cdot10^{-3}\) & \(5.2\cdot10^{-6}\) & \(4.2\cdot10^{-6}\) & \(3.6\cdot10^{-6}\) & \(4.5\cdot10^{-6}\) & \(2.5\cdot10^{-6}\) & \(4.8\cdot10^{-6}\) \\
%& \(1\cdot10^{-2}\) & \(3.7\cdot10^{-6}\) & \(4.2\cdot10^{-6}\) & \(3.6\cdot10^{-6}\) & \(3.8\cdot10^{-6}\) & \(2.5\cdot10^{-6}\) & \(4.8\cdot10^{-6}\) \\
%& \(1\cdot10^{-1}\) & \(3.7\cdot10^{-6}\) & \(4.2\cdot10^{-6}\) & \(3.6\cdot10^{-6}\) & \(3.8\cdot10^{-6}\) & \(2.5\cdot10^{-6}\) & \(4.8\cdot10^{-6}\) \\
\bottomrule
\end{tabular}
\end{table}

Considering Figure~\ref{fig:lin_vs_nonlind_dist} and Table~\ref{tab:correlation-transposed}, we observe that non-linear models consistently achieve lower intrinsic distances than their affine counterparts, i.e.,
\[
d_{\mathcal{F}_{\mathrm{Lip}_1}} < d_{\mathrm{Aff}},
\]
especially at moderate learning rates such as \( \text{lr} = 10^{-3} \).  
This indicates that the class of 1-Lipschitz neural networks \( \mathcal{F}_{\text{Lip}_1} \subset \mathcal{C}_{V,W} \) offers greater expressivity and better captures the structure-preserving mappings between encoder representations.

Interestingly, Spearman correlation \( \rho \) tends to be higher for affine transformations, suggesting that they preserve the rank ordering of feature dimensions more faithfully than their non-linear counterparts.  
This highlights a trade-off: non-linear models achieve tighter approximations in terms of distance, but may introduce distortions in the internal feature geometry.

This highlights a trade-off: non-linear models achieve tighter approximations in terms of distance, but may introduce distortions in the internal feature geometry.

As discussed earlier, the Pearson correlation between source and target representations remains close to zero across all tasks, underscoring the absence of any strong linear alignment between encoder representations.  
This further explains the limited performance of affine models and their inability to approximate a transformation across the models.


At high learning rates (e.g., \( \text{lr} = 10^{-1} \)), performance degrades in both model classes, with non-linear models exhibiting greater instability due to their higher complexity and non-convex loss landscape.  
Affine models remain more robust under such conditions but are fundamentally limited in their capacity to approximate complex transformations.



%In summary, non-linear transformations can yield significantly better alignment when properly optimized, especially at moderate learning rates. However, their performance is highly dependent on careful hyperparameter tuning. Affine models, while less expressive, provide a more stable baseline across all learning rates.


Now that we have gained a general understanding of the model behavior, we turn to the core concept of intrinsic homotopy.

To this end, we analyze two sets of heatmaps: one based on affine transformations (Figure~\ref{fig:lin_intrinsic_model}) and one on non-linear transformations (Figure~\ref{fig:nonlin_intrinsic_model}).  
Model pairs classified as intrinsically homotopic, i.e., with intrinsic distance below threshold, are highlighted with circles in both plots.

Since the \texttt{MRPC} task yielded the lowest overall approximation distances across learning rates, we focus our comparative analysis on this benchmark and restrict attention to the learning rates \(10^{-5}\), \(10^{-4}\), \(10^{-3}\), and \(10^{-2}\).

Recall that intrinsic homotopy is defined by the relation
\[
h \gtrsim_\mathrm{Intr} g \quad \Leftrightarrow \quad d_{\mathcal{F}_{\mathrm{Lip}_1}}(h, g) = 0,
\]
where \( d_{\mathcal{F}_{\mathrm{Lip}_1}} \) denotes the intrinsic distance.

In practice, however, the models operate on 768-dimensional hidden representations, and due to numerical and optimization limitations, exact zeros are generally not achievable.  
We therefore treat all distances below a threshold of \(1\) as effectively zero and consider such model pairs to be homotopic.

\begin{figure}[H]
	\centering
	\includegraphics[width=\linewidth]{Abschlussarbeit/Pictures/heatmaps_smaller_circles/Heatmap_linear_distance_all_lrs_mrpc_homotopy.png}
	\caption{Heatmaps of intrinsic distances for affine models on MRPC across learning rates. Pairs with intrinsic distance $<1$ are circled.}
	\label{fig:lin_intrinsic_model}
\end{figure}

\begin{figure}[H]
	\centering
	\includegraphics[width=\linewidth]{Abschlussarbeit/Pictures/heatmaps_smaller_circles/Heatmap_nonlinear_distance_all_lrs_mrpc_homotopy.png}
	\caption{Heatmaps of intrinsic distances for non-linear models on MRPC across learning rates. Pairs with intrinsic distance $<1$ are circled.}
	\label{fig:nonlin_intrinsic_model}
\end{figure}

Across all learning rates, we observe that non-linear transformations yield substantially more homotopic model pairs, especially at lower learning rates.  
This supports our previous findings from Figure~\ref{fig:lin_vs_nonlind_dist} and Table~\ref{tab:correlation-transposed}: non-linear models not only achieve smaller average distances but also induce denser and more coherent homotopy structures among encoders.

At higher learning rates, however, both model classes exhibit fewer such pairs, reflecting training instability and diminished approximation quality.  
This reinforces the earlier observation that effective representation alignment critically depends on proper learning rate tuning, particularly for complex transformation classes such as \( \mathcal{F}_{\text{Lip}_1} \).



\section{Implementierung und Experimente}
\frame{\tableofcontents[currentsection]}
\begingroup
\frametitle{Implementierung: Trainingsprozedur (nichtlinear)}
\begin{frame}
	\footnotesize
	\begin{itemize}
		\item \textbf{Ziel:} Lerne eine Transformation \( \psi \), sodass \( \psi(g(x)) \approx h(x) \) für Encoderrepräsentationen \( g, h \)
		\item \textbf{Modellklassen:}
		\begin{itemize}
			\item \texttt{LinearMap} (affine Projektion)
			\item \texttt{NonLinearNetwork} (1-Lipschitz-MLP via Spektralnorm)
		\end{itemize}
		\item \textbf{Evaluation:}
		\begin{itemize}
			\item \( \max\limits_{x \in \mathcal{D}_{\text{test}}} \| h(x) - \psi(g(x)) \|_\infty \)
			\item Pearson-/Spearman-Korrelation, Lipschitz-Konstante
		\end{itemize}
	\end{itemize}
	
\end{frame}
\endgroup

%\begingroup
%\frametitle{Train Nonlinear Network}
%\begin{frame}[fragile]
%	\begin{figure}
%		\includegraphics[width=0.8\linewidth]{BilderPräsentation/TrainNN.png}
%	\end{figure}
%\end{frame}
%\endgroup 

\begingroup
\frametitle{Reproduktion der Intrinsic Homotopy Werte}
\begin{frame}{Reproduktion der Intrinsic Homotopy Werte}
	\begin{columns}
	\column{0.4\textwidth}
	\begin{minipage}[t]{\linewidth}
		\vspace{1em}
		\begin{itemize}
			\tiny
			\uncover<1->{\item  Ziel: Reproduktion der Distanzwerte aus Chan et al.~\cite{chan_affine_2024}}
			\uncover<2->{\item Erster Versuch: Approximation der Werte im Testsplit, selber Architektur und $\max(\text{loss})$}
			\uncover<3->{\item Zweiter Versuch: Approximation der Werte mit im Trainingssplit, selber Architektur und $\max(\text{loss})$}
			\uncover<4->{\item Modelle sind aufgrund des Trainings auf den Trainingsdaten gebiased, daher erfolgen Tests auf einem separaten Testsplit}
		\end{itemize}
	\end{minipage}
	\column{0.6\textwidth}
\uncover<1->{	\includegraphics[width=\linewidth]{BilderPräsentation/New_all_in_one_scatter_praes.png}}
\end{columns}
\end{frame}
\endgroup


\begingroup
\frametitle{Experiment: Intrinsic Homotopy}
\begin{frame}
	\begin{columns}
		\column{0.4\textwidth}
		\begin{minipage}[t]{\linewidth}
			\vspace{1em}
			\tiny
			\begin{itemize}
				\tiny
				\uncover<1->{\item Analyse linearer und nicht-linearer Transformationen}
			\end{itemize}
		\end{minipage}
		
		\column{0.6\textwidth}
		\uncover<2->{\includegraphics[width=1.1\linewidth]{BilderPräsentation/intrinsic_distance_all_tasksPräsi.png}}
	\end{columns}
\end{frame}
\endgroup

\begingroup
\frametitle{Experiment: Intrinsic Homotopy}
\begin{frame}
			\vspace{1em}
			\tiny
			\begin{itemize}
				\tiny
				\uncover<1->{\item Analyse linearer und nicht-linearer Transformationen}
				\uncover<2->{\item Pearson-Koeffizient: zeigt linearen Zusammenhang zwischen Input und Output}
				\uncover<3->{\item Kaum linearer Zusammenhang $\Longrightarrow$ schwierig, lineare Transformation zu finden}
				\uncover<4->{\item Lineare Modelle reichen nicht aus, um Strukturähnlichkeit realistisch abzubilden}
			\end{itemize}
		\begin{table}[H]
			\centering
			\caption{Pearson Correlation Score \(r\) across tasks and learning rates.}
			\label{tab:correlation-transposed}
			\begin{tabular}{l c ccccccc}
				\toprule
				 & \texttt{QNLI} & \texttt{CoLA} & \texttt{MRPC} & \texttt{QQP} & \texttt{RTE} & \texttt{SST-2} \\
				\midrule
				\(r\) 
				&  \(5\cdot10^{-6}\) & \(4\cdot10^{-6}\) & \(4\cdot10^{-6}\) & \(5\cdot10^{-6}\) & \(3\cdot10^{-6}\) & \(5\cdot10^{-6}\)
			\end{tabular}
		\end{table}
\end{frame}
\endgroup




\begingroup
\frametitle{Experiment: Intrinsic Homotopy}
\begin{frame}
	\begin{columns}
		\column{0.4\textwidth}
		\begin{minipage}[t]{\linewidth}
			\vspace{1em}
			\tiny
			\begin{itemize}
				\tiny
				\uncover<1->{\item \textbf{Beispiel-Task:} \textbf{MRPC} – geringste Distanzen im Intrinsic-Vergleich}
				\uncover<2->{\item \textbf{Heatmaps:} Modelle mit $d_{\mathcal{C}}(h, g) < 1$  und  $d_{\mathcal{C}}(g, h) < 1$ gelten als \emph{homotop} (weiß markiert)}
				\uncover<3->{\item \textbf{Beobachtungen:}}
				\begin{itemize}
					\tiny
					\uncover<4->{\item Deutlich mehr homotope Modellpaare bei \textbf{nichtlinearen Transformationen}}
					\uncover<5->{\item Besonders bei kleinen Lernraten (\texttt{lr} = $10^{-5}$ bis $10^{-3}$)}
					\uncover<6->{\item Höhere Lernraten $\Rightarrow$ instabileres Training, weniger Struktur}
				\end{itemize}
			\end{itemize}
		\end{minipage}
		
		\column{0.6\textwidth}
		\uncover<2->{\includegraphics[width=\linewidth]{BilderPräsentation/PräsiHeatmap_linear_distance_all_lrs_mrpc_homotopy.png}\\[1ex]}
		\uncover<2->{\includegraphics[width=\linewidth]{BilderPräsentation/PräsiHeatmap_nonlinear_distance_all_lrs_mrpc_homotopy.png}}
	\end{columns}
\end{frame}
\endgroup


\begingroup
\frametitle{Experiment: Intrinsic vs Extrinsic Homotopy}
\begin{frame}
	\tiny
	\begin{itemize}
		\tiny
		\uncover<1->{\item Intrinsic $\Rightarrow$ Extrinsic in Chan et al.~\cite{chan_affine_2024}} 
		\uncover<2->{\item \textbf{Beobachtungen:}}
		\begin{itemize}
			\tiny
			\uncover<2->{\item Beste Übereinstimmung bei \textbf{QNLI} (höchste Regressionsgüte)}
			\uncover<2->{\item Aber selbst hier nur leichter zusammenhang }
		\end{itemize}
	\end{itemize}
	\vspace{1em}
	\uncover<2->{\includegraphics[width=1\linewidth]{BilderPräsentation/intrinsicVSextrinsic/PräsiDist_extr_intr_subplots_qnli.png}}
\end{frame}
\endgroup

\begingroup
\frametitle{Experiment: Intrinsic vs Extrinsic Homotopy}
\begin{frame}
	\tiny
	\begin{itemize}
		\tiny
		\uncover<1->{\item \textbf{Beobachtungen:}}
		\begin{itemize}
			\tiny
			\uncover<1->{\item Regressionslinie bei \textbf{MRPC} nahezu parallel zur X-Achse }
			\uncover<1->{\item $\Rightarrow$ Kaum Zusammenhang zwischen intrinsischer und extrinsischer Ähnlichkeit in unserem Setup, stark vom Task abhängig}
		\end{itemize}
	\end{itemize}
	\vspace{1em}
		\uncover<1->{\includegraphics[width=1\linewidth]{BilderPräsentation/intrinsicVSextrinsic/PräsiDist_extr_intr_subplots_mrpc.png}}
\end{frame}
\endgroup


\section{Zusammenfassung und Future Work}
\frame{\tableofcontents[currentsection,currentsubsection]}
\begingroup
\frametitle{Fazit}
\begin{frame}
	\begin{itemize}
		\setlength\itemsep{0.6em}
		\uncover<1->{\item Nichtlineare Homotopie erlaubt tiefere Vergleiche von Sprachmodellen als affine Baselines.}
		\uncover<2->{\item Besonders bei kleinen Lernraten entstehen dichte, homotope Strukturen zwischen Encodern.}
		\uncover<3->{\item Zusammenhang zwischen intrinsischer und extrinsischer Homotopie ist stark \textbf{task-abhängig} (z.\,B. QNLI vs. SST-2).}
	\end{itemize}
\end{frame}
\endgroup

\begingroup
\frametitle{Ausblick}
\begin{frame}
	\begin{itemize}
		\setlength\itemsep{0.9em}
		\uncover<1->{\item \textbf{Modellvielfalt:} Erweiterung auf andere Modellarten.}
		\uncover<2->{\item \textbf{IT-Sicherheit:} Anwendung im Bereich Poisoning- oder Backdoor-Angriffe.}


	\end{itemize}
\end{frame}
\endgroup



\appendix\nocite{*}
\section{Literatur}
\begin{frame}[allowframebreaks, nosection]
	\frametitle{Literatur}
	\bibliography{quellen}

\end{frame}

\begin{frame}[nosection]
    Vielen Dank für Ihre Aufmerksamkeit!
\end{frame}


\end{document}