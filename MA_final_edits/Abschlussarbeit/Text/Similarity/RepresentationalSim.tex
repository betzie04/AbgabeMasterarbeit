%Within this section, we define the \textit{representational measure similarity}, that measures the difference between the activations of intermediate layers.
%Afterwards, we give an overview about representational similarity of neural networks which is determined by a similarity measure $m$. 

In order to compare neural networks, representational similarity methods typically assess how similar the internal representations of a fixed set of inputs are across different models or layers.  
We will focus in the following on comparing activations at specific layers, but alternative approaches may also consider other components, such as weights or gradients.


%In order to compare neural networks, representational similarity measurements examine how similar the activations of a fixed set of inputs at any pair of layers are.
Therefore, we consider a representation of the model $f$ at layer $l$:
\begin{equation*}
    R:=R^{(l)}=(f^{(l)} \circ f^{(l-1)} \circ ... \circ f^{(1)}) \in \mathbb{R}^{N\times D}.
\end{equation*}
Here, \( N \) denotes the (possibly infinite) number of datapoints, and \( D \) the number of neurons in a single layer.


\begin{definition}[Representational similarity measures] A representational similarity measure is a map 
\begin{equation*}
    m: \mathbb{R}^{N\times D} \times \mathbb{R}^{N\times D} \longrightarrow \mathbb{R},
\end{equation*} 
that assigns a similarity score $m(R, R')$ to a pair of representations $R,R'$ of different models $f,f'$ that use the same inputs $\mathcal{X}$.    
\end{definition}

Nevertheless, we can consider two representation matrices \( R, R' \in \mathbb{R}^{N \times D} \) to be equivalent, even if they are not identical element-wise.
This is particularly relevant when comparing language encoders whose outputs may differ due to arbitrary linear transformations, such as sign flips (e.g., \( R = -R' \)) or rotations.
Such transformations do not necessarily alter the underlying structure or information content of the representations.
Hence, comparing representations up to such equivalence classes allows us to focus on their functional behavior rather than their raw numerical values.

The specific type of transformation used to define equivalence depends on the context and intended application.  
For instance, some analyses may permit any orthogonal transformation, while others restrict equivalence to affine or even task-informed transformations.

%Nevertheless, we can consider two representation matrices $R,R'\in \mathbf{R}^{N\times D}$ to be equivalent, even if they are not identical on an element-per-element basis.
%This is the case, if we consider representations that differ in their sign i.e. $R=-R'$, or when we rotate onto another. 

By using terms of bijective mappings (transformations), we can formalize the notions of equivalence as follows:
\begin{align*}
    \psi:\mathbb{R}^{N\times D} &\to \mathbb{R}^{N\times D}\\
    \psi(R)&=R'.
\end{align*}


\begin{definition}[Equivalence of Representations]
    Let $R,R' \in\mathbb{R}^{N\times D}$ be two representations and $\mathcal{T}:=\mathcal{T}(N,D)$ a group. 
    We denote $R$ and $R'$ as equivalent with respect to a group $\mathcal{T}$, written as $R \sim_\mathcal{T}R'$ if there is a $\psi\in \mathcal{T}$, such that $\psi(R)=R'$
\end{definition}
The group of transformations under which representations should be considered equivalent is of crucial importance in practice, since equivalent representations should be indistinguishable for the chosen similarity measure.
In contrast, non-equivalent representations must be distinguishable for a similarity measure.
Formally, this means that a measure must be invariant to exactly those groups of transformations under which the underlying representations are equivalent.

%\begin{definition}[Invariant Representational Similarity Measure]
%    Let $R,R' \in\mathbb{R}^{N\times D}$ be two representations, $\mathcal{T}(N,D)$ a group and $\varphi\in \mathcal{T}(N,D), \varphi'\in \mathcal{T}(N,D)$.
%    A similarity measure $m$ is called invariant to a group of transformations $\mathcal{T}$ if it holds:
%    \[m(R,R')=m(\varphi(R),\varphi'(R)) \quad  \forall R,R' \text{ and } \forall\varphi, \varphi'\]  
%    \[\Longrightarrow m(R,R)=m(R,R') \text{ if } R \sim_\mathcal{T} R'\]
%\end{definition}

\begin{definition}[Invariant Representational Similarity Measure]
Let $\mathcal{T}(N, D)$ be a group of transformations acting on representation matrices in $\mathbb{R}^{N \times D}$.  
A similarity measure \( m: \mathbb{R}^{N \times D} \times \mathbb{R}^{N \times D} \to \mathbb{R} \) is called \emph{invariant under} \( \mathcal{T}(N, D) \) if for all \( R, R' \in \mathbb{R}^{N \times D} \) and all \( \varphi, \varphi' \in \mathcal{T}(N, D) \), it holds that
\[
    m(R, R') = m(\varphi(R), \varphi'(R')).
\]
In particular, if \( R \sim_{\mathcal{T}} R' \), then \( m(R, R) = m(R, R') \).
\end{definition}

Using this definition, we will later examine whether the similarity measures considered satisfy the axioms of a distance metric.  
In the context of representational similarity, however, one of these axioms—namely the \emph{identity of indiscernibles}, can be relaxed.  
Rather than requiring that \( m(R, R') = 0 \) if and only if \( R = R' \), it is sufficient to require that
\[
m(R, R') = 0 \iff R \sim_{\mathcal{T}} R',
\]
where \( \mathcal{T} \) is a group of transformations under which the measure \( m \) is invariant~\cite{williams_generalized_2021}.  
This allows the similarity measure to capture equivalence of representations up to transformations that preserve their functional or structural meaning.




%Using this definition, we can analyse later in the thesis, whether the measures meet the criteria of a distance metric.
%In the sense of representational similarity the criteria can be relaxed to only require $m(R,R')= 0$ if and only if $R\sim_\mathcal{T}R'$ for a group of transformations $\mathcal{T}$ that $m$ is invariant to \cite{williams_generalized_2021}.



%An instance representation $R_i=((f^{(l)} \circ f^{(l-1)} \circ ... \circ f^{(1)})(\mathbf{X_i}\in \mathbb{R}^D$ represents the activations of instances $i \in \{1,...,N\}$, which corresponds to the $i$-th row. of $R$