In this chapter we formalize the mathematical methods of our novel approach.
We focus on continuous non-linear transformations instead of affine ones, as presented by Chan et al. \cite{chan_affine_2024}.
The non-linear transformations are represented by a neural network in practice, as it is one of the most used state-of-the-art approaches for non-linear transformations.
We will formalize the mathematical methods to analyse the influence of continuous non-linear transformations on the approach presented in the chapter \ref{EA}.
First, we will modify the definition of intrinsic and extrinsic homotopy. 
%Therefore, we consider the set of continuous maps, which includes the set of affine maps and the set of non-linear maps, and extend these homotopies to a set of continuous maps.
Therefore, we extend the notion of homotopy by considering the set of continuous maps, which includes both linear and non-linear functions, as long as they are continuous.
Furthermore, we define an approach, where the similarity will be measured by the models performance. 


\section{Intrinsic Homotopy on Non-linear Transformations}\label{InrinsicHomotopyonNonlinTransf}
Inspired by the framework of Chan et al.~\cite{chan_affine_2024}, we consider a space of language encoders \( \mathcal{E}_V \), viewed as a subset of the space of continuous functions from sequences \( \Sigma^* \) to a finite-dimensional vector space \( V \subset \mathbb{R}^d \), where \( \Sigma \) denotes a finite alphabet and \( \Sigma^* \) its Kleene closure.

%Inspired by the framework of Chan et al. \cite{chan_affine_2024}, we consider a space of language encoders \( \mathcal{E}_V \), viewed as a subset of the space of continuous functions from sequences \( \Sigma^* \) to a finite-dimensional vector space \( V \subset \mathbb{R}^d \). 
Equipped with the uniform norm
\[
\|h\|_\infty := \sup_{x \in \Sigma^*} |h(x)|_V,
\]
the space \( (\mathcal{E}_V, \|\cdot\|_\infty) \) becomes a complete metric space with distance \( d_\infty(h, g) := \|h - g\|_\infty \), due to the completeness of \( V \).

The goal of this section is to define an intrinsic, nonlinear notion of homotopy between language encoders.  
While Chan et al. \cite{chan_affine_2024} define such a structure using affine maps, we extend this approach to a more expressive class of transformations: namely, general nonlinear maps.  
In principle, we are interested in the full range of nonlinear structure-preserving transformations.  
However, in order to ensure that the induced distance satisfies the triangle inequality, and thus defines a valid hemi-metric, we restrict ourselves to 1-Lipschitz continuous functions.  
This technical constraint is essential in the proof of Proposition~\ref{eq:intrinsic_equation}, where we rely on the fact that 1-Lipschitz functions are closed under composition and do not increase the \( \ell^\infty \)-distance.


Intuitively, we say that an encoder \( h \) is intrinsically related to another encoder \( g \) if there exists a well-behaved transformation that maps \( g \) closely to \( h \). 
To formalize this idea, we define an asymmetric notion of distance \( d_{\mathcal{F}_{\text{Lip}_1}}(h, g) \), which quantifies how well \( h \) can be approximated by transformed versions of \( g \) under a class of admissible maps. The smaller this distance, the more intrinsically related the models are. If the distance is zero, we consider the two encoders intrinsically similar.

We now make this precise using the notion of a hemi-metric as introduced in Definition~\ref{def:Hemi_Metr_Func_Space}. 

Let \( \mathcal{F} \subset \mathcal{C}(V, V) \) denote a class of admissible transformations, where \( V \subset \mathbb{R}^d \) is a finite-dimensional vector space.  
Each transformation \( \psi \in \mathcal{F} \) maps encoder outputs from \( V \) into \( V \), and induces a new encoder \( \psi \circ g \colon \Sigma^* \to V \).

In the following, we will focus on the case where \( \mathcal{F} = \mathcal{F}_{\mathrm{Lip}_1} \), the set of 1-Lipschitz continuous functions from \( V \) to itself.


Then, for any pair of encoders \( h,  g \in \mathcal{E}_V \), we define the hemi-metric, as in Definition ~\ref{def:Hemi_Metr_Func_Space} %directed distance
\[
d_{ \mathcal{F}_{\mathrm{Lip}_1}}(h, g) := \inf_{\psi \in \mathcal{F}_{\mathrm{Lip}_1}} \| h - \psi \circ g \|_\infty,
\]
which measures how well \( h \) can be approximated by transformed versions of \( g \). 
This construction induces a preorder \( h \gtrsim_{\mathrm{Intr}} g \) on the space of encoders.
 
\begin{definition}[Intrinsic Relation]\label{eq:intrinsic_equation}
Let \( h, g \in \mathcal{E}_V \), and let \(  \mathcal{F}_{\mathrm{Lip}_1} \subset \mathcal{C}(V, V) \) be a class of admissible transformations.
We say that \( h \) is intrinsically related to \( g \), written
\[
h \gtrsim_{\mathrm{Intr}} g,
\]
if the induced distance satisfies
\[
d_{\mathcal{F}_{\text{Lip}_1}}(h, g) = \inf_{\psi \in \mathcal{F}_{\mathrm{Lip}_1}} \| h - \psi \circ g \|_\infty = 0.
\]
\end{definition}

As we need to restrict ourselves to 1-Lipschitz continuous functions, we will introduce the Lipschitz Continuty in the following.   

\begin{definition}[Lipschitz Continuity]
	A function \( \psi \colon V \to W \) between two normed vector spaces is called \emph{Lipschitz continuous} \cite{forster_analysis_2025} if there exists a constant \( L \geq 0 \) such that for all \( v_1, v_2 \in V \), we have
	\[
	\| \psi(v_1) - \psi(v_2) \|_W \leq L \cdot \| v_1 - v_2 \|_V.
	\]
	The smallest such constant \( L \) is called the \emph{Lipschitz constant} of \( \psi \). If \( L \leq 1 \), we say that \( \psi \) is \emph{1-Lipschitz}.
\end{definition}


\begin{proposition}\label{prop:complete_F_Lip}
Let \(  \mathcal{F}_{\mathrm{Lip}_1} \subset \mathcal{C}(V, V) \) denote the set of 1-Lipschitz continuous functions from \( V \) to \( W \), where \( V \) and \( W \) are finite-dimensional normed vector spaces over \( \mathbb{R} \).
Then \( \mathcal{F}_{\text{Lip}_1} \) is a closed subset of the complete metric space \( (\mathcal{C}(V, V),d_{\mathcal{F}_{\text{Lip}_1}}) \), and hence itself complete.
\end{proposition}

\begin{proof}
Let \( (\psi_n) \subset \mathcal{F}_{\text{Lip}_1} \) be a sequence that converges uniformly to some function \( \psi \in \mathcal{C}(V, V) \) with respect to the supremum norm
\[
\|\psi_n - \psi\|_\infty := \sup_{x \in V} \|\psi_n(x) - \psi(x)\|_W.
\]

We show that \( \psi \) is also 1-Lipschitz:
for any \( x, y \in V \),
\[
\|\psi(x) - \psi(y)\|_W \leq \|\psi(x) - \psi_n(x)\|_W + \|\psi_n(x) - \psi_n(y)\|_W + \|\psi_n(y) - \psi(y)\|_W.
\]
Using the triangle inequality and the fact that \( \psi_n \in \mathcal{F}_{\text{Lip}_1} \), we get
\[
\|\psi(x) - \psi(y)\|_W \leq 2\|\psi - \psi_n\|_\infty + \|x - y\|_V.
\]
Taking the limit \( n \to \infty \), the first term vanishes and we obtain
\[
\|\psi(x) - \psi(y)\|_W \leq \|x - y\|_V,
\]
so \( \psi \) is 1-Lipschitz and thus \( \psi \in \mathcal{F}_{\text{Lip}_1} \). Therefore, \( \mathcal{F}_{\text{Lip}_1} \) is closed.

Since \( (\mathcal{C}(V, V), d_\infty) \) is a complete metric space (by completeness of \( W \) \cite{werner_funktionalanalysis_2005}), it follows that \( \mathcal{F}_{\text{Lip}_1} \) is complete as a closed subset of a complete space.
\end{proof}



To formalize how well an encoder \( h \) can be approximated by another encoder \( g \) via a transformation from a given function class, we define the \emph{intrinsic approximation distance}
\[
d_{\mathcal{F}_{\text{Lip}_1}}(h, g) := \inf_{\psi \in  \mathcal{F}_{\mathrm{Lip}_1}} \| h - \psi \circ g \|_\infty,
\]
where \(\mathcal{F}_{\mathrm{Lip}_1}\subset \mathcal{C}(V, V) \) denotes a set of admissible transformations – for instance, the class of 1-Lipschitz continuous maps.
This defines a \emph{hemi-metric} on the space of encoders \( \mathcal{E}_V \), as it may fail to be symmetric.

The \emph{specialization quasi-ordering} induced by this hemi-metric is given by
\[
h \gtrsim_{ \mathcal{F}_{\mathrm{Lip}_1}} g \quad \Leftrightarrow \quad d_{\mathcal{F}_{\text{Lip}_1}}(h, g) = 0.
\]
Intuitively, this means that \( h \) can be approximated arbitrarily well by applying some transformation \( \psi \in  \mathcal{F}_{\mathrm{Lip}_1} \) to \( g \). That is, for any error tolerance \( \varepsilon > 0 \), there exists a transformation \( \psi \in \mathcal{F}_{\mathrm{Lip}_1} \) such that \( \| h - \psi \circ g \|_\infty < \varepsilon \). The relation \( \gtrsim_{ \mathcal{F}_{\mathrm{Lip}_1}} \) is thus a \emph{preorder}: it is reflexive and transitive, but not necessarily antisymmetric.

%\begin{proposition}[Intrinsic Preorder]
%Let \( \mathcal{F}_{\text{Lip}_1} \subset \mathcal{C}(V, V) \) denote the class of 1-Lipschitz continuous transformations from \( V \) to itself.
%Then the relation \( \gtrsim_{\mathrm{Intr}} \) defined by
%\[
%h \gtrsim_{\mathrm{Intr}} g \quad \Leftrightarrow \quad \inf_{\psi \in \mathcal{F}_{\text{Lip}_1}} \| h - \psi \circ g \|_\infty = 0
%\]
%is a preorder on \( \mathcal{E}_V \); that is, it is reflexive and transitive.
%\end{proposition}

\begin{proposition}[Intrinsic Preorder]
	Let \( \mathcal{F}_{\text{Lip}_1} \subset \mathcal{C}(V, V) \) denote the class of 1-Lipschitz continuous functions from \( V \) to itself, and define the map
	\[
	d_{\mathcal{F}_{\text{Lip}_1}} : \mathcal{E}_V \times \mathcal{E}_V \to \mathbb{R}, \quad (h, g) \mapsto \inf_{\psi \in \mathcal{F}_{\text{Lip}_1}} \| h - \psi \circ g \|_\infty.
	\]
	Then \( d_{\mathcal{F}_{\text{Lip}_1}} \) is a hemi-metric, and the induced specialization quasi-ordering
	\[
	h \gtrsim_{ \mathcal{F}_{\mathrm{Lip}_1}} g \quad \Longleftrightarrow \quad d_{\mathcal{F}_{\text{Lip}_1}}(h, g) = 0
	\]
	defines a preorder on \( \mathcal{E}_V \), by Lemma~\ref{lemma:quasi-ordering}.
\end{proposition}


\begin{proof}
	We verify that \(  d_{\mathcal{F}_{\mathrm{Lip}_1}} \) is a hemi-metric on \( \mathcal{E}_V \), as required by the axioms of Definition~\ref{def:hemi_metric}.
	
	\begin{itemize}
		\item[H1:]
		For any \( h \in \mathcal{E}_V \), we have \( \mathrm{id} \in  \mathcal{F}_{\mathrm{Lip}_1} \), so
		\[
		d_{ \mathcal{F}_{\mathrm{Lip}_1}}(h, h) \leq \| h - \mathrm{id} \circ h \|_\infty = 0.
		\]
		Since the infimum is non-negative, it follows that \( 	d_{ \mathcal{F}_{\mathrm{Lip}_1}}(h, h) = 0 \).
		
		\item[H2:] Let \( h, g, f \in \mathcal{E}_V \). We want to show:
		\[
			d_{ \mathcal{F}_{\mathrm{Lip}_1}}(h,f) \leq 	d_{ \mathcal{F}_{\mathrm{Lip}_1}}(h,g) + 	d_{ \mathcal{F}_{\mathrm{Lip}_1}}(g,f).
		\]
		
		Since \( \mathcal{F}_{\text{Lip}_1} \subset \mathcal{C}(V, V) \) is the set of 1-Lipschitz continuous maps acting on the complete metric space \( (\mathcal{E}_V, d_\infty) \), we can approximate the infimum arbitrarily well.
		
		That is, for any \( \varepsilon > 0 \), there exist alignments:
		\begin{itemize}
			\item \( \psi_g \in \mathcal{F}_{\text{Lip}_1} \) such that
			\[
			\| h - \psi_g(g) \|_\infty \leq 	d_{ \mathcal{F}_{\mathrm{Lip}_1}}(h, g) + \varepsilon,
			\]
			\item \( \psi_f \in \mathcal{F}_{\text{Lip}_1} \) such that
			\[
			\| g - \psi_f(f) \|_\infty \leq 	d_{ \mathcal{F}_{\mathrm{Lip}_1}}(g, f) + \varepsilon.
			\]
		\end{itemize}
		
		Such approximations exist because the infimum in \( 	d_{ \mathcal{F}_{\mathrm{Lip}_1}} \) is taken over a set of continuous functions, and \( \mathcal{E}_V \) is complete. In particular, minimizing sequences \( (\psi_n) \subset \mathcal{F}_{\text{Lip}_1} \) with \( \| h - \psi_n(g) \|_\infty \to	d_{ \mathcal{F}_{\mathrm{Lip}_1}}(h, g) \) converge (up to subsequences) to a map in \( \mathcal{F}_{\text{Lip}_1} \), so approximations up to \( \varepsilon \) are always available.
		
		Since \( \mathcal{F}_{\text{Lip}_1} \) is closed under composition (composition of 1-Lipschitz maps is again 1-Lipschitz), we can define
		\[
		\psi := \psi_g \circ \psi_f \in \mathcal{F}_{\text{Lip}_1}.
		\]
		Then
		\[
			d_{ \mathcal{F}_{\mathrm{Lip}_1}}(h, f) \leq \| h - \psi(f) \|_\infty = \| h - \psi_g(\psi_f(f)) \|_\infty.
		\]
		
		We estimate this norm using the triangle inequality:
		\[
		\| h - \psi_g(\psi_f(f)) \|_\infty 
		\leq \| h - \psi_g(g) \|_\infty + \| \psi_g(g) - \psi_g(\psi_f(f)) \|_\infty.
		\]
		
		The first term is bounded by \(	d_{\mathcal{F}_{\mathrm{Lip}_1}}(h, g) + \varepsilon \) by construction. To bound the second term, we use the fact that \( \psi_g \) is 1-Lipschitz, i.e., has Lipschitz constant \( L = 1 \):
		\[
		\| \psi_g(g) - \psi_g(\psi_f(f)) \|_\infty \leq L \cdot \| g - \psi_f(f) \|_\infty \leq 	d_{\mathcal{F}_{\mathrm{Lip}_1}}(g, f) + \varepsilon.
		\]
		
		Putting it all together:
		\[
		d_{\mathcal{F}_{\mathrm{Lip}_1}}(h, f) \leq \| h - \psi_g(g) \|_\infty + \| g - \psi_f(f) \|_\infty 
		\leq d_{\mathcal{F}_{\mathrm{Lip}_1}}(h, g) +  d_{\mathcal{F}_{\mathrm{Lip}_1}}(g, f) + 2\varepsilon.
		\]
		
		Since \( \varepsilon > 0 \) was arbitrary, the triangle inequality follows:
		\[
		 d_{\mathcal{F}_{\mathrm{Lip}_1}}(h, f) \leq  d_{\mathcal{F}_{\mathrm{Lip}_1}}(h, g) +  d_{\mathcal{F}_{\mathrm{Lip}_1}}(g, f).
		\]
		
	\end{itemize}

	
	Note that the triangle inequality in this form relies on the fact that all transformation functions in \( \mathcal{F}_{\text{Lip}_1} \) have Lipschitz constant \( L = 1 \).  
	If we allowed larger constants \( L > 1 \), the second term 
	\[
	\| \psi_g(g) - \psi_g(\psi_f(f)) \|_\infty \leq L \cdot \| g - \psi_f(f) \|_\infty
	\]
	would scale the second distance term by \(L\), and we would only obtain the weaker inequality
	\[
	 d_{\mathcal{F}_{\mathrm{Lip}_1}}(h, f) \leq  d_{\mathcal{F}_{\mathrm{Lip}_1}}(h, g) + L \cdot  d_{\mathcal{F}_{\mathrm{Lip}_1}}(g, f),
	\]
	which does not satisfy the standard triangle inequality unless \(L = 1\).  
	Hence, the restriction to 1-Lipschitz maps is essential to ensure that \(  d_{\mathcal{F}_{\mathrm{Lip}_1}} \) defines a hemi-metric.
	
	Consequently, \(  d_{\mathcal{F}_{\mathrm{Lip}_1}} \) satisfies the axioms of a hemi-metric on \( \mathcal{E}_V \) in the sense of Definition~\ref{def:hemi_metric}.
	
	To complete the proof, we verify that the relation
	\[
	h \gtrsim_{\mathrm{Intr}} g \quad \Longleftrightarrow \quad  d_{\mathcal{F}_{\mathrm{Lip}_1}}(h, g) = 0
	\]
	is a preorder on \( \mathcal{E}_V \):
	
	\begin{itemize}
		\item Reflexivity: For all \( h \in \mathcal{E}_V \), we have \(  d_{\mathcal{F}_{\mathrm{Lip}_1}}(h, h) = 0 \) by axiom (H1), so \( h \gtrsim_{\mathrm{Intr}} h \).
		\item Transitivity: Let \( h \gtrsim_{\mathrm{Intr}} g \) and \( g \gtrsim_{\mathrm{Intr}} f \), i.e., \(  d_{\mathcal{F}_{\mathrm{Lip}_1}}(h, g) = 0 \) and \(  d_{\mathcal{F}_{\mathrm{Lip}_1}}(g, f) = 0 \). Then by the triangle inequality:
		\[
		 d_{\mathcal{F}_{\mathrm{Lip}_1}}(h, f) \leq  d_{\mathcal{F}_{\mathrm{Lip}_1}}(h, g) +  d_{\mathcal{F}_{\mathrm{Lip}_1}}(g, f) = 0,
		\]
		hence \(  d_{\mathcal{F}_{\mathrm{Lip}_1}}(h, f) = 0 \), so \( h \gtrsim_{\mathrm{Intr}} f \).
	\end{itemize}
	
	Thus, \( \gtrsim_{\mathrm{Intr}} \) is reflexive and transitive, i.e., a preorder.
\end{proof}


\begin{definition}[Exact Intrinsic Non-Linear Homotopy]
Let \( h, g \in \mathcal{E}_V \), and let \( \mathcal{F}_{\text{Lip}_1} \subset \mathcal{C}(V, V) \) denote the class of admissible 1-Lipschitz continuous transformations.  
We say that \( h \) and \( g \) are \emph{exactly intrinsically non-linearly homotopic}, written
\[
h \gtrsim_{\mathcal{F}_{\text{Lip}_1}} g,
\]
if both directional distances vanish:
\[
d_{\mathcal{F}_{\text{Lip}_1}}(h, g) := \inf_{\psi \in \mathcal{F}_{\text{Lip}_1}} \| h - \psi \circ g \|_\infty = 0
\quad \text{and} \quad
d_{\mathcal{F}_{\text{Lip}_1}}(g, h) := \inf_{\phi \in \mathcal{F}_{\text{Lip}_1}} \| g - \phi \circ h \|_\infty = 0.
\]
That is, each encoder can be approximated arbitrarily well (in the \( \ell^\infty \)-norm) by applying admissible transformations to the other.
\end{definition}




\section{Extrinsic Homotopy on Non-linear Transformations}
\label{ExtrinsicHomotopyonNonlinTransf}
In this section, we extend the notion of homotopy to incorporate extrinsic structure, focusing not only on the encoder representations themselves, but also on their behavior when composed with downstream classifiers.

Whereas intrinsic homotopy compares encoder functions directly via transformations in function space, extrinsic homotopy evaluates the functional output of composite systems, such as encoder–classifier pipelines.

The key idea is to assess whether one encoder can emulate another by applying suitable post-processing transformations e.g., neural classifiers, followed by normalization.

To formalize this, let \( V \subset \mathbb{R}^d \) denote the output space of an encoder and \( W = \mathbb{R}^C \) the target space of classifier outputs (e.g., logits).
We fix a class of admissible transformations \( \mathcal{C}_{V,W} \subset \mathcal{C}(V,W) \), consisting of continuous, potentially non-linear maps that preserve relevant structure (e.g., Lipschitz continuity or smoothness). 
These transformations model classifiers applied to encoder outputs.

Let $\Sigma$ be a finite alphabet and let $\Sigma^*$ denote the set of all finite strings over $\Sigma$, i.e., the Kleene closure of $\Sigma$.  
Let $V$ be a vector space of intermediate representations, and let $W := \mathbb{R}^C$ denote the output space for a classification task with $C$ classes.

We fix a class of admissible transformations \( \mathcal{C}_{V,W} \subset \mathcal{C}(V, W) \), consisting of continuous, potentially nonlinear functions (e.g., neural networks) that map encoder representations into classifier logits. These functions model downstream classifiers.

We define the following sets of functions:
\begin{itemize}
	\item $\mathcal{E}_V := \operatorname{Map}(\Sigma^*, V)$ – encoders mapping strings to vector representations,
	\item $\mathcal{E}_W := \operatorname{Map}(\Sigma^*, W)$ – classifiers mapping strings to class scores (logits),
	\item $\mathcal{E}_{\Delta^{C-1}} := \operatorname{Map}(\Sigma^*, \Delta^{C-1})$ – classifiers returning class probability distributions.
\end{itemize}

Given an encoder $h \in \mathcal{E}_V$, we define the set of classifier functions applied to $h$ as
\[
\mathcal{C}_{V,W}(h) := \left\{ \psi \circ h \mid \psi \in \mathcal{C}_{V,W} \right\} \subset \mathcal{E}_W.
\]
Applying the softmax function with inverse temperature $\lambda > 0$ yields the associated family of nonlinear logit-to-probability classifiers:
\[
\mathcal{V}_C(h) := \left\{ \operatorname{softmax}_\lambda \circ f \mid f \in \mathcal{C}_{V,W}(h) \right\} \subset \mathcal{E}_{\Delta^{C-1}}.
\]
Here, $\operatorname{softmax}_\lambda: W \to \Delta^{C-1}$ is defined as
\[
\operatorname{softmax}_\lambda(x)_i := \frac{\exp(\lambda x_i)}{\sum_{j=1}^C \exp(\lambda x_j)} \quad \text{for } x \in \mathbb{R}^C,\; i = 1,\dots,C.
\]


This framework allows us to compare encoders extrinsically by measuring how closely the outputs of their composed pipelines align. 
Following Definition~\ref{def:Hemi_Metr_Func_Space}, we define the extrinsic hemi-metric as:
%\[
%d^{\mathcal{H}}_{\mathcal{C}_{V,W}}(h, h) := \sup_{\psi \in \mathcal{C}_{V,W}} \inf_{\varphi \in \mathcal{C}_{V,W}} \| \operatorname{softmax}_\lambda ( \psi \circ g) - \operatorname{softmax}_\lambda( \varphi \circ h )\|_\infty,
%\]
\[
d^{\mathcal{H}}_{\mathcal{C}_{V,W}}(h, g) := \sup_{\psi \in \mathcal{C}_{V,W}} \inf_{\varphi \in \mathcal{C}_{V,W}} \|  \psi \circ h - \varphi \circ g \|_\infty,
\]

which quantifies how well the output of encoder \( g \) can be aligned to that of \( h \) through admissible post-transformations in \( \mathcal{C}_{V,W} \).  


\begin{definition}[Norm-induced Distance Functions]
Using the hemi-metric framework introduced in Definition~\ref{def:Hemi_Metr_Func_Space}, we define two norm-induced distance functions on the encoder space \( \mathcal{E}_V \):

\begin{align*}
d^\mathcal{H}_{\mathcal{C}_{V,W}}(h,g) &:= d^\mathcal{H}_{\infty, W}\left( \mathcal{C}_{V,W}(h), \mathcal{C}_{V,W}(g) \right), \\
d_{\mathcal{F}_{\mathrm{Lip}_1}}^{\mathcal{H}}(h,g) &:= d^\mathcal{H}_{\infty, \Delta^{C-1}}\left( \mathcal{V}_C(h), \mathcal{V}_C(g) \right).
\end{align*}
\end{definition}
The first distance \( d^\mathcal{H}_{\mathcal{C}_{V,W}} \) compares the representations of \( h \) and \( g \) after all admissible nonlinear transformations, and quantifies how closely \( g \) can approximate \( h \) via such transformations.

The second distance \( d_{\mathcal{F}_{\mathrm{Lip}_1}}^{\mathcal{H}} \) measures how differently two encoders behave under downstream classification tasks, modeled via post-composition with a classifier followed by softmax normalization.

Both distances capture extrinsic discrepancies: the former in terms of raw feature transformations, the latter in terms of final classification behavior.


\begin{proposition}[Triangle Inequality for \( d^\mathcal{H}_{\mathcal{C}_{V,W}} \)]
Let \( h, g, f \in \mathcal{E}_V \). 
Then the following triangle inequality holds:
\[
d^\mathcal{H}_{\mathcal{C}_{V,W}}(h,f) \leq d^\mathcal{H}_{\mathcal{C}_{V,W}}(h,g) + d^\mathcal{H}_{\mathcal{C}_{V,W}}(g,f).
\]
\end{proposition}


\begin{proof}
We prove the inequality using the definition of \( d^\mathcal{H}_{\mathcal{C}_{V,W}} \) and the triangle inequality of the \( \infty \)-norm.

Let \( \epsilon > 0 \) be arbitrary.
Fix any \( \psi_h \in \mathcal{C}_{V,W} \). 

By the definition of the infimum, there exists \( \psi_g \in \mathcal{C}_{V,W} \) such that:
\[
d_{\infty,W}(\psi_h \circ h, \psi_g \circ g) \leq \inf_{\tilde{\psi}_g \in \mathcal{C}_{V,W}} d_{\infty,W}(\psi_h \circ h, \tilde{\psi}_g \circ g) + \epsilon.
\]
Likewise, for this fixed \( \psi_g \), there exists \( \psi_f \in \mathcal{C}_{V,W} \) such that:
\[
d_{\infty,W}(\psi_g \circ g, \psi_f \circ f) \leq \inf_{\tilde{\psi}_f \in \mathcal{C}_{V,W}} d_{\infty,W}(\psi_g \circ g, \tilde{\psi}_f \circ f) + \epsilon.
\]

Applying the triangle inequality of \( d_{\infty,W} \), we obtain:
\begin{align*}
d_{\infty,W}(\psi_h \circ h, \psi_f \circ f) 
&\leq d_{\infty,W}(\psi_h \circ h, \psi_g \circ g) + d_{\infty,W}(\psi_g \circ g, \psi_f \circ f) \\
&\leq \inf_{\tilde{\psi}_g \in \mathcal{C}_{V,W}} d_{\infty,W}(\psi_h \circ h, \tilde{\psi}_g \circ g) + \inf_{\tilde{\psi}_f \in \mathcal{C}_{V,W}} d_{\infty,W}(\psi_g \circ g, \tilde{\psi}_f \circ f) + 2\epsilon.
\end{align*}

Now take the infimum over \( \psi_f \) on the left-hand side and then the supremum over all \( \psi_h \in \mathcal{C}_{V,W} \) to obtain:
%\begin{align*}
%d^\mathcal{H}_{\mathcal{C}_{V,W}}(h,f) 
%&= \sup_{\psi_h \in \mathcal{C}_{V,W}} \inf_{\psi_f \in \mathcal{C}_{V,W}} d_{\infty,W}(\psi_h \circ h, \psi_f \circ f) \\
%&\leq \sup_{\psi_h \in \mathcal{C}_{V,W}} \left( \inf_{\tilde{\psi}_g \in \mathcal{C}_{V,W}} d_{\infty,W}(\psi_h \circ h, \tilde{\psi}_g \circ g) + \inf_{\tilde{\psi}_f \in \mathcal{C}_{V,W}} d_{\infty,W}(\psi_g \circ g, \tilde{\psi}_f \circ f) \right) + 2\epsilon \\
%&\leq d^\mathcal{H}_{\mathcal{C}_{V,W}}(h,g) + d^\mathcal{H}_{\mathcal{C}_{V,W}}(g,f) + %2\epsilon.
%\end{align*}

\begin{align*}
&d^\mathcal{H}_{\mathcal{C}_{V,W}}(h,f) 
= \sup_{\psi_h \in \mathcal{C}_{V,W}} \inf_{\psi_f \in \mathcal{C}_{V,W}} d_{\infty,W}(\psi_h \circ h, \psi_f \circ f) \\
&\leq \sup_{\psi_h \in \mathcal{C}_{V,W}} \left( \inf_{\tilde{\psi}_g \in \mathcal{C}_{V,W}} d_{\infty,W}(\psi_h \circ h, \tilde{\psi}_g \circ g) + \inf_{\tilde{\psi}_f \in \mathcal{C}_{V,W}} d_{\infty,W}(\tilde{\psi}_g \circ g, \tilde{\psi}_f \circ f) \right) + 2\epsilon \\
&\leq \sup_{\psi_h \in \mathcal{C}_{V,W}} \inf_{\tilde{\psi}_g \in \mathcal{C}_{V,W}} d_{\infty,W}(\psi_h \circ h, \tilde{\psi}_g \circ g) 
+ \sup_{\tilde{\psi}_g \in \mathcal{C}_{V,W}} \inf_{\tilde{\psi}_f \in \mathcal{C}_{V,W}} d_{\infty,W}(\tilde{\psi}_g \circ g, \tilde{\psi}_f \circ f) 
+ 2\epsilon \\
&= d^\mathcal{H}_{\mathcal{C}_{V,W}}(h,g) + d^\mathcal{H}_{\mathcal{C}_{V,W}}(g,f) + 2\epsilon.
\end{align*}


Since \( \epsilon > 0 \) was arbitrary, the result follows.
\end{proof}



\begin{proposition}\label{prop:softmax_lip}
The temperature-scaled softmax function
\[
\operatorname{softmax}_\lambda(z)_i := \frac{e^{\lambda z_i}}{\sum_{j=1}^n e^{\lambda z_j}}
\]
is Lipschitz continuous with respect to the $\ell^\infty$ norm with Lipschitz constant at most $\frac{\lambda}{2}$.
\end{proposition}

\begin{proof}
The softmax function is continuously differentiable, and its Jacobian matrix \\ \(J(z) \in \mathbb{R}^{n \times n}\) has entries
\[
J_{ij}(z) = \lambda \cdot s_i \cdot (\delta_{ij} - s_j),
\quad \text{where } s_i = \operatorname{softmax}_\lambda(z)_i.
\]
The row sums of the absolute values are
\[
\sum_{j=1}^n |J_{ij}(z)| = \lambda \cdot s_i \left( |1 - s_i| + \sum_{j \ne i} s_j \right)
= 2\lambda \cdot s_i(1 - s_i).
\]
The product $s_i(1 - s_i)$ is maximized at $s_i = \frac{1}{2}$, hence
\[
\sum_j |J_{ij}(z)| \le \frac{\lambda}{2} \quad \text{for all } i.
\]
Therefore, the operator norm satisfies $\|J(z)\|_\infty \le \frac{\lambda}{2}$, and the function $\operatorname{softmax}_\lambda$ is \\ $\frac{\lambda}{2}$~-Lipschitz with respect to the $\ell^\infty$ norm.
\end{proof}


%\begin{proposition}[Extrinsic Preorder]
	%Let \( \mathcal{F}_{\text{Lip}_1} \subset \mathcal{C}(V, W) \) denote the class of 1-Lipschitz continuous functions from \( V \) to itself, and define the map
	%\[
	%d_{\mathcal{F}_{\text{Lip}_1}} : \mathcal{E}_V \times \mathcal{E}_V \to \mathbb{R}, \quad (h, g) \mapsto \inf_{\psi \in \mathcal{F}_{\text{Lip}_1}} \| h - \psi \circ g \|_\infty.
	%\]
	%Then \( d_{\mathcal{F}_{\text{Lip}_1}} \) is a hemi-metric, and the induced specialization quasi-ordering
	%\[
	%h\gtrsim_{\mathcal{C}_\text{Ext}} g \quad \Longleftrightarrow \quad d_{\mathcal{F}_{\text{Lip}_1}}(h, g) = 0
	%\]
	%defines a preorder on \( \mathcal{E}_V \), by Lemma~\ref{lemma:quasi-ordering}.
%\end{proposition}


\begin{definition}[Extrinsic Relation]
	Let $h, g \in \mathcal{E}_V$. 
	We define the extrinsic relation as:
	\[
	h \gtrsim_{\mathrm{Ext}} g \quad \Leftrightarrow \quad d_{\mathcal{F}_{\mathrm{Lip}_1}}^{\mathcal{H}}(h,g) = 0.
	\]
\end{definition}

\begin{proposition}[Extrinsic Preorder]
	Let \( \mathcal{F}_{\mathrm{Lip}_1} \subset \mathcal{C}(V, W) \) denote the class of 1-Lipschitz continuous functions from \( V \) to \( W \).  
	Define the extrinsic approximation distance by
	\[
	d_{\mathcal{F}_{\mathrm{Lip}_1}}^{\mathcal{H}}(h, g) := \sup_{\psi \in \mathcal{F}_{\mathrm{Lip}_1}} \inf_{\varphi \in \mathcal{F}_{\mathrm{Lip}_1}} \| \psi \circ h - \varphi \circ g \|_\infty,
	\]
	for \( h, g \in \mathcal{E}_V \), where \( \mathcal{E}_V \subset \operatorname{Map}(\Sigma^*, V) \) is the space of encoders into \( V \).
	
	Then \( d_{\mathcal{F}_{\mathrm{Lip}_1}}^{\mathcal{H}} \) is a hemi-metric on \( \mathcal{E}_V \), and the induced relation
	\[
	h \gtrsim_{\mathrm{Ext}} g \quad \Longleftrightarrow \quad d_{\mathcal{F}_{\mathrm{Lip}_1}}^{\mathcal{H}}(h, g) = 0
	\]
	defines a preorder on \( \mathcal{E}_V \).
\end{proposition}





\begin{proof}
We show first, that $d_{\mathcal{F}_{\mathrm{Lip}_1}}^{\mathcal{H}}$ is a hemi-metric as defined in Definition~\ref{def:hemi_metric}. 
%This implies that the induced relation is a preorder.


\begin{itemize}
    \item [H1:] $d_{\mathcal{F}_{\mathrm{Lip}_1}}^{\mathcal{H}} \geq 0$.
    \item[] Since $d_{\mathcal{F}_{\mathrm{Lip}_1}}^{\mathcal{H}}$ is a metric, the Hausdorff-Hoare-map inherits non-negativity
    \item [] $d_{\mathcal{F}_{\mathrm{Lip}_1}}^{\mathcal{H}}(h,h) =  0 \quad \forall h \in \mathcal{E}_V$.
    %\item \textbf{Identity of Indiscernible} $d_{\mathcal{F}_{\mathrm{Lip}_1}}^{\mathcal{H}}(h,h) =  0 \quad \forall h \in \mathcal{E}_V$.
    \item[] For $h=g$ we have $\mathcal{C}_{V,W}(h)=\mathcal{V}(V,\Delta)(g)$ and the infimum becomes zero.
    \item [H2:] Let \( h, g, f \in \mathcal{E}_V \).
    We show that
    %\item \textbf{Triangle Inequality:} Let \( h, g, f \in \mathcal{E}_V \). We show that
    \[
    d_{\mathcal{F}_{\mathrm{Lip}_1}}^{\mathcal{H}}(h,f) \leq d_{\mathcal{F}_{\mathrm{Lip}_1}}^{\mathcal{H}}(h,g) + d_{\mathcal{F}_{\mathrm{Lip}_1}}^{\mathcal{H}}(g,f).
    \]
    To do so, consider the following setup: Let \( h \in \mathcal{E}_V \) be a language encoder \( h: \Sigma^* \to V \), and let \( \psi \in \mathcal{C}_{V,W} \) be a transformation \( \psi: V \to W \). We define the composed map
    \[ p_\psi := \operatorname{softmax}_\lambda(\psi \circ h ): \Sigma^* \to \Delta^{C-1},
    \]
    which can be interpreted as assigning probability distributions over \( C \) classes. This yields the compositional structure
    \[
    \Sigma^* \xrightarrow{h} V \xrightarrow{\psi} W \xrightarrow{\operatorname{softmax}_\lambda} \Delta^{C-1}.
    \]

   % Since \( \operatorname{softmax}_\lambda \) is a Lipschitz-continuous map with constant \( L = \frac{1}{\lambda} \), the Lipschitz property carries over to the composition with any function \( \psi \in \mathcal{C}_{V,W} \). 
    
Since Proposition~\ref{prop:softmax_lip} shows, \( \operatorname{softmax}_\lambda \colon \mathbb{R}^N \to \Delta^{N-1} \) is differentiable with bounded Jacobian entries, it is Lipschitz continuous with respect to the \( \ell^\infty \)-norm. In particular, there exists a constant \( L > 0 \) such that for all \( x, y \in \mathbb{R}^N \),
\[
\|\operatorname{softmax}_\lambda(x) - \operatorname{softmax}_\lambda(y)\|_\infty \leq L \cdot \|x - y\|_\infty.
\]
This Lipschitz continuity extends to function composition: for any \( \psi \in \mathcal{C}_{V,W} \), the composed function \( \operatorname{softmax}_\lambda( \psi) \in \mathcal{C}_{V,\Delta^{C-1}} \) satisfies
\[
\| \operatorname{softmax}_\lambda(\psi(h)) - \operatorname{softmax}_\lambda(\psi(g)) \|_\infty \leq L \cdot \| \psi(h) - \psi(g) \|_\infty.
\]

In particular, the Hausdorff–Hoare distances satisfy
\[
d_{\infty, \Delta^{C-1}}^\mathcal{H}(\operatorname{softmax}_\lambda( \mathcal{C}_{V,W}(h)), \operatorname{softmax}_\lambda (\mathcal{C}_{V,W}(g)) )
\leq L \cdot d^\mathcal{H}_{\infty, W}(\mathcal{C}_{V,W}(h), \mathcal{C}_{V,W}(g)),
\]
where the Hausdorff–Hoare distance is computed with respect to the supremums norm over \( V \) and the \( \ell^\infty \)-norm on the codomain.

As a result, the triangle inequality for \( d^\mathcal{H}_{\mathcal{C}_{V,W}} \) implies the triangle inequality for the softmax-composed version,
\[
d_{\mathcal{F}_{\mathrm{Lip}_1}}^{\mathcal{H}}(h,f) \leq d_{\mathcal{F}_{\mathrm{Lip}_1}}^{\mathcal{H}}(h,g) + d_{\mathcal{F}_{\mathrm{Lip}_1}}^{\mathcal{H}}(g,f),
\]
provided that \( L = 1 \).  
If \( L > 1 \), the inequality remains valid after rescaling the distance function accordingly.

    
%    In particular, the Hausdorff–Hoare distances satisfy:
%    \[
%    d_{\infty, \Delta^{C-1}}^\mathcal{H}(\operatorname{softmax }_\lambda( \mathcal{ C}_{V,W}(h)), \operatorname{softmax}_\lambda (\mathcal{C}_{V,W}(g)) )\leq L \cdot d^\mathcal{H}_{\infty, W}(\mathcal{C}_{V,W}(h), \mathcal{C}_{V,W}(g)).
%    \]

    %As a result, the triangle inequality for \( d^\mathcal{H}_{\mathcal{C}_{V,W}} \) implies the triangle inequality for the softmax-composed version:
    %\[
    %d_{\mathcal{F}_{\mathrm{Lip}_1}}^{\mathcal{H}}(h,f) \leq d_{\mathcal{F}_{\mathrm{Lip}_1}}^{\mathcal{H}}(h,g) + d_{\mathcal{F}_{\mathrm{Lip}_1}}^{\mathcal{H}}(g,f)
    %\]
    %provided \( \operatorname{softmax}_\lambda \) is 1-Lipschitz or the constants are appropriately absorbed.
\end{itemize}

Hence, \( d_{\mathcal{F}_{\mathrm{Lip}_1}}^{\mathcal{H}} \) is a hemi-metric.%, and the induced relation \( \gtrsim_{\mathcal{C}_\text{Ext}} \) is a preorder.

%Now define a binary relation \( \gtrsim_d \) on \( \mathcal{E}_V \) by
%\[
%h \gtrsim_d g \quad \Leftrightarrow \quad d(h, g) = 0.
%\]

We show that this relation $\gtrsim_{\mathrm{Ext}}$ is a preorder:
\begin{itemize}
    \item[]Reflexivity: For all \( h \in \mathcal{E}_V \), \( d_{\mathcal{F}_{\mathrm{Lip}_1}}^{\mathcal{H}}(h, h) = 0 \), so \( h \gtrsim_{\mathrm{Ext}} h \).
    \item[]Transitivity: If \( h \gtrsim_{\mathrm{Ext}} g \) and \( g \gtrsim_{\mathrm{Ext}} f \), then
    \[
    d_{\mathcal{F}_{\mathrm{Lip}_1}}^{\mathcal{H}}(h, f) \le d_{\mathcal{F}_{\mathrm{Lip}_1}}^{\mathcal{H}}(h, g) + d_{\mathcal{F}_{\mathrm{Lip}_1}}^{\mathcal{H}}(g, f) = 0 + 0 = 0,
    \]
    so \( h \gtrsim_{\mathrm{Ext}} f \).
\end{itemize}
Based on Lemma~\ref{lemma:quasi-ordering}, the relation \( \gtrsim_{\mathrm{Ext}} \) defined by
\[
h \gtrsim_{\mathrm{Ext}} g \quad \Longleftrightarrow \quad d_{\mathcal{F}_{\mathrm{Lip}_1}}^{\mathcal{H}}(h, g) = 0
\]
is a preorder on the encoder space \( \mathcal{E}_V \).  
This coincides with our definition of the extrinsic approximation relation \( \gtrsim_{\mathrm{Ext}} \),  
as both are defined by the vanishing of the approximation distance \( d_{\mathcal{F}_{\text{Lip}_1}} \).

\end{proof}

%\begin{definition}[Extrinsic Preorder]
%For two encoders \( h, g \in \mathcal{E}_V \), we define the relation
%\[
%h \gtrsim_{\mathcal{C}_\text{Ext}} g \quad \Leftrightarrow \quad %d_{\mathcal{F}_{\mathrm{Lip}_1}}^{\mathcal{H}}(h,g) = 0.
%\]
%\end{definition}

\begin{definition}[Exact Extrinsic Nonlinear Homotopy]
An encoder \( h \in \mathcal{E}_V \) is said to be \emph{exactly extrinsically non-linearly homotopic to} \( g \in \mathcal{E}_V \), written
\[
h \gtrsim_\text{Ext} g,
\]
if
\[
d_{\mathcal{F}_{\mathrm{Lip}_1}}^{\mathcal{H}}(h,g) = 0.
\]
\end{definition}

Note that this relation is generally not symmetric; it only expresses that \( h \) can be approximated arbitrarily well by applying admissible transformations to \( g \), not necessarily the other way around.
This directional behavior reflects the asymmetric notion of homotopy proposed in \cite{chan_affine_2024}, where similarity is understood in terms of functional approximation from one model to another.


\section{Performance-Based Similarity Measure}\label{sec:PBSM}

As explained in Chapter~\ref{SoNN}, functional similarity measures compare the outputs \( \mathbf{O}, \mathbf{O}' \in \mathbb{R}^{N\times C} \), where each element \( \mathbf{O}_{i,c} \) denotes the predicted probability or score for class \( c \) on input \( x_i \). The predicted class label is given by
\[
\hat{y}_i = \arg \max_c \mathbf{O}_{i,c}.
\]

In this work, we focus exclusively on \emph{classification agreement} as a performance-based similarity measure. This captures how often two models assign the same predicted label to the same input, independent of confidence scores or loss values.

Let \( \mathbf{O}^{(h)} \) and \( \mathbf{O}^{(g)} \) be the class score matrices of models \( h \) and \( g \), respectively. Their predicted labels are defined as
\[
\hat{y}^{(h)}_i = \arg\max_c \mathbf{O}^{(h)}_{i,c}, \quad
\hat{y}^{(g)}_i = \arg\max_c \mathbf{O}^{(g)}_{i,c}.
\]

The agreement score is then computed as
\[
\operatorname{Agreement}(h, g) = \frac{1}{N} \sum_{i=1}^N \mathds{1} \left[ \hat{y}^{(h)}_i = \hat{y}^{(g)}_i \right],
\]
where \( \mathds{1} \) denotes the indicator function.

This simple yet interpretable metric reflects functional alignment in terms of model decisions and is well suited for classification tasks.


%In this work, we focus on \emph{Spearman's rank correlation} as a performance-based similarity measure. It evaluates how well two models agree in the relative ordering of class scores for each input, regardless of the absolute values.

%\paragraph{Spearman's Rank Correlation}
%Spearman's rank correlation measures the strength and direction of a monotonic relationship between two ranked variables~\cite{sammut_encyclopedia_2017}.
%Given ranks \( x_i \) and \( y_i \) for $N$ paired observations, it is computed as:
%\[
%\rho = 1 - \frac{6 \sum_{i=1}^N (x_i - y_i)^2}{N(N^2 - 1)},
%\]
%assuming no tied ranks. In practice, ties can occur, especially with discrete prediction scores or limited numerical precision.
%In such cases, a corrected version of the Spearman correlation should be used, e.g., via Pearson correlation applied to the ranks. 
%Most statistical libraries apply these corrections automatically.

%In the context of model comparison, Spearman's correlation is computed between the class score vectors predicted by two models for each input.
%It captures the consistency in confidence structures and is especially useful when the ranking of predictions matters more than their absolute magnitude.
