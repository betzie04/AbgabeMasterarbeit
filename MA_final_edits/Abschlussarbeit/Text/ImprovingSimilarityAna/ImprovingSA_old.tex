Within this chapter, we will formalize the mathematical methods, to analyse the influence of continuous nonlinear transformations on the approach, introduced in chapter \ref{EA}.
First, we will modify the definition of intrinsic homotopy.
Afterwards, the definition of extrinsic homotopy is redefined. 
Therefore, we will consider the set of continuous maps, which includes the set of affine maps and the set nonlinear maps and extend those homotopies on a set of continuous maps.
Further, as Chan et al. \cite{chan_affine_2024} already presented an approach for affine maps.
We will focus on nonlinear maps, which are represented by a neural network in practice, as it is one of the most used state of the art approaches for nonlinear transformations.


\section{Intrinsic Homotopy on nonlinear Transformations}
As Chan et al. \cite{chan_affine_2024} we consider $\mathcal{E}_V$, the vectorspace of language encoders.

To redefine an intrinsic homotopy, we use definition \ref{def:Hemi_Metr_Func_Space} to introduce a map from one encoder to another one, using a continuous transformation.
%As nonlinear transformations are a subset of continuous maps, we will first consider the set of continuous maps to define a preorder. 
Based on the required properties of definition \ref{def:Hemi_Metr_Func_Space}, we consider a subset of the set of continuous maps $S\subset \mathcal{C}(V)$ and as previously defined \[d_S(h,g):=d_\infty^\mathcal{H}(h,S(g)):= \inf_{\psi \in S} \|h-\psi(g)\|_\infty,\] where $\psi \in \mathcal{C}(V)$. 

Let's remember the defined preorder in section \ref{IH}, as an reflexive and transitive relation. 

\begin{definition}[Intrinsic Continuous Preorder] For two encoder $h,g \in \mathcal{E}_V$, we define the relation:
\[  h \gtrsim_\mathcal{C} g \quad \Leftrightarrow \quad d_\mathcal{C}(h,g)=0 \]
\end{definition}

Further, we need to prove, that the relation $h \gtrsim_{\text{cnt}} g$ is a preorder on $\mathcal{E}_V$.

Therefore, we verify first, that $d_\mathcal{C}$ is a hemi-metric on $\mathcal{E}_V$ and confirm the axioms from section \ref{HMS}.\\
\begin{itemize}
    \item [1.] \textbf{Non-negativity:} $d_\mathcal{C}(h,g)\geq 0$
    \item [] By definition $\|\cdot\|_\infty$ is a norm, so $\inf\|h-\psi(g)\|_\infty\geq0$ for any $\psi$ and $h,g\in \mathcal{E}_V$
    \item [2.] \textbf{Identity of indiscernible:} $d_\mathcal{C}(h,g)= 0$
    \item [] For $h=g$, chose $\psi=id$. Then:\[
    d_\mathcal{C}(h,g) = \inf\|h-\psi(g)\|_\infty\leq\|h-id(g)\|_\infty = 0.
    \]
    Since $d_\mathcal{C}(h,g)\geq 0$, we have $d_\mathcal{C}(h,g) = 0$.
    \item [3.] \textbf{Triangle inequality:} Let $h,g,f\in\mathcal{E}_V$.
    We want to proof:
    \[d_\mathcal{C}(h,f) \leq d_\mathcal{C}(h,g)+d_\mathcal{C}(g,f)\]
    \item [] Let $\psi_1$ denote the alignment from $g$ to $h$ and $\psi_2$ the alignment from $f$ to $g$ and the composition of $\psi_1\circ \psi_2$ aligns $f$ to h:
    \[
        d_\mathcal{C}(h,f)\leq \|h-\psi_1\circ \psi_2(f)\|_\infty\leq \underbrace{\|h-\psi_1(g)\|_\infty}_\text{bounded by $d_\mathcal{C}(h,g)$} +\underbrace{\|\psi_1(g)-\psi_1\circ\psi_2(f)\|_\infty}_\text{not directly related to $d_\mathcal{C}(g,f)$}.
    \]
       To get a relation to $d_\mathcal{C}(g,f)$ consider $\psi_1$ as Lipschitz map.
    Then it holds:
    \[\|\psi_1(g)-\psi_1\circ\psi_2(f)\|_\infty \leq L \cdot\|g-\psi_2(f)\|_\infty\leq L\cdot d_\mathcal{C}(g,f)\]
    \[\Longrightarrow \quad  d_\mathcal{C}(h,f)\leq  d_\mathcal{C}(h,g) + L\cdot d_\mathcal{C}(g,f) \underbrace{\leq}_{L \leq 1}  d_\mathcal{C}(h,g) +  d_\mathcal{C}(g,f).\]
\end{itemize}

Consequently, $d_\mathcal{C}(h,g) := \inf_{\psi\in\text{Lip}_1}\|h-\psi(g)\|_\infty$ and the compact metric space $\mathcal{E}_V$ form a hemi-metric space. 


\section{Extrinsic Homotopy on nonlinear Transformations}

Further, we redefine extrinsic homotopy on a family of nonlinear models and nonlinear transformations.
    
Let $W$ be the vector space $\mathbb{R}^N$ and set $\mathcal{C}_{V,W}$ as set of all nonlinear maps from $V$ to $W$. 
Consequently, $\mathcal{E}_W$ denotes the vector space of continuously transformed language encoder, which represents a map from the Kleene Closure into the vector space $W$.
We set $\mathcal{E}_{\Delta^{N-1}}$ as map from the Kleene Closure into the $N-1$ dimensional probability simplex represented by $\Delta^{N-1}$.
A transfer learning task as construction a classifier that useses language encoders string can be formalized.
Therefore, we set $\mathcal{V}_N$ to be the family of nonlinear-models and formalize the notion as follows:

\begin{align*}
\mathcal{V}_N \colon \mathcal{E}_V &\to \mathcal{P}(\mathcal{E}_{\Delta^{N-1}}) \setminus \{\emptyset\}, \\
h &\mapsto \operatorname{softmax}_\lambda(\mathcal{C}_{V,W}(h)), \\
\end{align*}
Where $h \in \mathcal{E}_V$ and $\operatorname{softmax}$ as defined in the previous section.


$\mathcal{C}_{V,W}$ denotes the map:

\begin{align*}
\mathcal{C}_{V,W}:\mathcal{E}_V &\to \mathcal{P(E_\text{W}})\setminus\{\emptyset\}
\\
h &\mapsto \{\psi\circ h \quad | \quad \psi \in \mathcal{C}_{V,W}\}
\end{align*}

Using definition \ref{def:Hemi_Metr_Func_Space} we define the following hemi-metrics on $\mathcal{E}_V$.

\begin{align*}
d^\mathcal{H}_{\mathcal{C}(V,W)}(h,g):=d^\mathcal{H}_{\infty,W}(\mathcal{C}_{V,W}(h), \mathcal{C}_{V,W}(g))\\
d^\mathcal{H}_{\mathcal{V}(V,\Delta)}:= d^\mathcal{H}_{\infty,\Delta^{N-1}}(\mathcal{V}_N(h),\mathcal{V}_N(g)).
\end{align*}

The map $d^\mathcal{H}_{\mathcal{C}(V,W)}$ compares all possible affine transformations of two encoder, by answering how badly does the best nonlinear transformation of $g$ fail to match the transinformation of $h$.
The map $d^\mathcal{H}_{\mathcal{V}(V,\Delta)}$ considers how different two encoder behave on any downstream classification tast after a nonlinear transformation. k
%As we want to measure how close we can bring $h$ and $g$ if we transform both nonlinear, we rather use $d^\mathcal{H}$, as Chan et al. \cite{chan_affine_2024} did. 
%This corresponds to independently transform the encoder nonlinear on the same transfer learning task. 

As above, we want to define a extrinsic preorder using $d^\mathcal{H}_{\mathcal{V}(V,\Delta)}$.

\begin{definition}[Extrinsic Continuous Preorder]
For two encoder $h,g\in \mathcal{E}_V$, we define the relation 
\[h \gtrsim_{\mathcal{C}_\text{Ext}} g \quad \Leftrightarrow \quad d^\mathcal{H}_{\mathcal{V}(V,\Delta)}(h,g)=0. \]
\end{definition}

Further, we need to prove, that the relation $h \gtrsim_{\mathcal{C}_\text{Ext}} g$ is a preorder on $\mathcal{E}_V.$

As in the previous, we verify that $d^\mathcal{H}_{\mathcal{C}_{V,W}} := \sup_{x\in V}\inf_{y\in W}d(x,y)$ is a hemi-metric on $\mathcal{E}_V$.

\begin{enumerate}
    \item \textbf{Non-negativity:} $d^\mathcal{H}_{\mathcal{C}_{V,W}} \geq 0$.
    \item[] Since $d_{\infty,W}$ is a metric, the Hausdorff-Hoare-map inherits non-negativity
    \item \textbf{Identity of indiscernible} $d^\mathcal{H}_{\mathcal{C}_{V,W}}(h,h) =  0 \quad \forall h \in \mathcal{E}_V$. 
    \item[] For $h=g$ we have $\mathcal{C}_{V,W}(h)=\mathcal{C}_{V,W}(g)$ and the infimum becomes zero.
    \item \textbf{Triangle inequality} Let $h,g,f \in \mathcal{E}_V$. We aim to proof:
     \[d^\mathcal{H}_{\mathcal{C}_{V,W}}(h,f) \leq d^\mathcal{H}_{\mathcal{C}_{V,W}}(h,g) + d^\mathcal{H}_{\mathcal{C}_{V,W}}(g,f)\]
     \item[]
\end{enumerate}

To prove the triangle inequailty, we construct an approximate map between $h\to g$ and $g \to f$. 
Further, we combine them using triangle inequality of the $\infty$-distance.
By using an $\epsilon>0$ we can approximate the infimum via an arbitrary closeness. 

Based on the definition \ref{def:hausdorffHoareMap} of the Hausdorff-Hoare map, 
\[d^\mathcal{H}_{\mathcal{C}_{V,W}}(h,f):= \sup_{\psi_h\in \mathcal{C}_{V,W}}\inf_{\psi_f\in \mathcal{C}_{V,W}}d_{\infty,W}(\psi_h\circ h, \psi_f\circ f),\] where $d_{\infty,W}$ is the distance \[d_{\infty,W}(u,v)=\sup_{y\in\Sigma^*}\|\psi_h\circ h(y)-\psi_f\circ f(y)\|_W\]


Let $\epsilon > 0$.
Based on the definition of  supreme and infimum, by definition of the infimum, for every $\epsilon>0$ and $\psi_h \in \mathcal{C}_{V,W}$, there exists a map $\psi_g\in \mathcal{C}_{V,W}$, such that
\[d_{\infty,W}(\psi_h\circ h, \psi_g \circ g)\leq \inf_{\tilde{\psi}_g\in \mathcal{C}_{V,W}}d_{\infty,W}(\psi_h\circ h, \tilde{\psi}_g\circ g) +\epsilon.\]
As well as, there exists a map $\psi_f\in\mathcal{C}_{V,W}$ for this $\psi_g$ where
\[d_{\infty, W}(\psi_g\circ g, \psi_f \circ f) \leq \inf_{\tilde{\psi}_f\in \mathcal{C}_{V,W}}d_{\infty,W}(\psi_g\circ g, \tilde{\psi}_f\circ f) +\epsilon.\]

Because of $d_{\infty, W}$ definition and the triange inequality it holds: 
Since the second term $\inf_{\tilde{\psi}_f\in \mathcal{C}_{V,W}}d_{\infty,W}(\psi_g\circ g, \tilde{\psi}_f\circ f)$ depends on $\psi_g$ which depends on $\psi_h$, we can upper-bound this by taing the supremum over all $\psi_g\in \mathcal{C}_{V,W}$.
We can justifie this by subadditivity of the supremum:
\[\sup_x(f(x) +g(x))\leq \sup_x f(x) + \sup_y g(y)\]
\[d_{\infty, W}(\psi_h\circ h, \psi_f \circ f)\leq d_{\infty, W}(\psi_h\circ h, \psi_g \circ g) +d_{\infty, W}(\psi_g\circ g, \psi_f \circ f).\]

As $\psi_f$ is chosen depending on $\psi_g$, which depends on $\psi_h$ we can upper bound $\psi_f$; too.

Consequently, it follows:
\[d_{\infty, W}(\psi_h\circ h, \psi_f \circ f)\leq \left(\inf_{\tilde{\psi}_g\in \mathcal{C}_{V,W}}(d_{\infty, W}(\psi_h\circ h, \tilde{\psi}_g \circ g) + \epsilon\right)+\left(\inf_{\tilde{\psi}_f\in \mathcal{C}_{V,W}}d_{\infty, W}(\psi_g\circ g, \tilde{\psi}_f \circ f)+\epsilon\right).\]

\[\Longrightarrow d_{\infty, W}(\psi_h\circ h, \psi_f \circ f)\leq \inf_{\tilde{\psi}_g\in \mathcal{C}_{V,W}}d_{\infty, W}(\psi_h\circ h, \tilde{\psi}_g \circ g)+\inf_{\tilde{\psi}_f\in \mathcal{C}_{V,W}}d_{\infty, W}(\psi_g\circ g, \tilde{\psi}_f \circ f)+2\epsilon.\]
Now consider the supremum on all $\psi_h\in \mathcal{C}_{V,W}$

\[\sup_{\psi_h\in\mathcal{C}_{V,W}} \inf_{\psi_f\in\mathcal{C}_{V,W}} d_{\infty, W}(\psi_h\circ h, \psi_f \circ f)\leq \]\[\left(\sup_{\psi_h\in\mathcal{C}_{V,W}}\inf_{\tilde{\psi}_g\in \mathcal{C}_{V,W}}d_{\infty, W}(\psi_h\circ h, \tilde{\psi}_g \circ g)+\inf_{\tilde{\psi}_f\in \mathcal{C}_{V,W}}d_{\infty, W}(\psi_g\circ g, \tilde{\psi}_f \circ f)+2\epsilon\right).\]
Because of monotonies and the upper claims we get: 

\[\sup_{\psi_h\in\mathcal{C}_{V,W}} \inf_{\psi_f\in\mathcal{C}_{V,W}} d_{\infty, W}(\psi_h\circ h, \psi_f \circ f)\leq \]\[\sup_{\psi_h\in\mathcal{C}_{V,W}}\inf_{\tilde{\psi}_g\in \mathcal{C}_{V,W}}d_{\infty, W}(\psi_h\circ h, \tilde{\psi}_g \circ g)+\sup_{\psi_h\in\mathcal{C}_{V,W}}\inf_{\tilde{\psi}_f\in \mathcal{C}_{V,W}}d_{\infty, W}(\psi_g\circ g, \tilde{\psi}_f \circ f)+2\epsilon.\]

Consequently it holds:
\[d^\mathcal{H}_{\mathcal{C}_{V,W}}(h,f) \leq d^\mathcal{H}_{\mathcal{C}_{V,W}}(h,g) + d^\mathcal{H}_{\mathcal{C}_{V,W}}(g,f)+2\epsilon\]

For $\epsilon \to 0$ we get:
$d^\mathcal{H}_{\mathcal{C}_{V,W}}(h,f) \leq d^\mathcal{H}_{\mathcal{C}_{V,W}}(h,g) + d^\mathcal{H}_{\mathcal{C}_{V,W}}(g,f)$


As the Hausdorff-Hoare map $d_{\mathcal{C}_{V,W}}^\mathcal{H}$ satisfies the required hemi-metric properties, it is proved, that it is a hemi-metric on $\mathcal{E}_V$

Further, let us consider the softmax-transformed image in the probability simplex $\Delta^{N-1}$ i.e.

\[\mathcal{V}_N(g) := \text{softmax}_\lambda(\mathcal{C}_{V,W}(h)) \subset \mathcal{E}_{\Delta^{N-1}},\]
and the hemi-metric $d^\mathcal{H}_{\mathcal{V}(V,\Delta)}:= d^\mathcal{H}_{\infty,\Delta^{N-1}}(\mathcal{V}_N(h),\mathcal{V}_N(g))$ as defined above.

We will use the general results of Lipschitz stability of the Hausdorff metric under Lipschitz maps.


Let $h\in\mathcal{E}_V$ be a language encoder $h:\Sigma^*\to V$ and let $\psi\in\mathcal{C}_{V,W}$ be a continuous map $\psi: V \to W$.
We define the composed map
\begin{align*}
    &p_\psi:\Sigma^*\to \Delta^{N-1}\\
    &p_\psi:=\text{softmax}_\lambda\circ \psi(h(y)),
\end{align*}
that can be seen as a probability distribution over $N$. 
It yields the composition:
\[p_\psi:\Sigma^*\xrightarrow{h} V \xrightarrow{\psi} W \xrightarrow{\text{softmax}_\lambda} \Delta^{N-1}\]

Let us consider the Lipschitz continuous map $T:= \text{softmax}_\lambda:W \to \Delta^{N-1}$ with constant $L$.

We can apply the fact of Lipschitz continuous maps with constant $L$.
For any function sets $h,g \in \mathcal{E}_V$ the Hausdorff-Hoare map satisfies:
\[d_{\infty,\Delta^{N-1}}^\mathcal{H}(\text{softmax}(\psi\circ h), \mathcal{V}_N(g))\leq L\cdot d_{\infty, W}^\mathcal{H}(\mathcal{C}_{V,W}(h),\mathcal{C}_{V,W}(g)).\]
Consequently, the triangle inequation holds for all $1-$Lipschitz continuous maps if the family of neural networks $\mathcal{V}_N$ is $1-$Lipschitz continuous.


%Now, we use that the Hausdorff-Hoare metric is non-expansive under Lipschitz maps.
%\[d_{\infty,\Delta^{N-1}}(\mathcal{V}_N\circ\mathcal{C}_{V,W}(h), \mathcal{V}_N\circ\mathcal{C}_{V,W}(f))\leq L\cdot d_{\infty,W}(\mathcal{C}_{V,W}(h), \mathcal{C}_{V,W}(f))\]



\section{Relation of Intrinsic and Extrinsic Homotopy}
Within this section, we will focus on the relation of intrinsic and extrinsic homotopy.
Therefore, we relate $d_\mathcal{C}(h,g)$ and $d_{\mathcal{C}_{V,W}}^\mathcal{H}$
