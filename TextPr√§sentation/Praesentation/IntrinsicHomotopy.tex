
%\begin{frame}
%	\textbf{Was ist Intrinsic Homotopy?}\\
%	\begin{itemize}
%		\item Nichtlineare Transformation zwischen Repräsentationen zweier Modelle
%		\item Idee: Modell A kann in Modell B deformiert werden
%		\item Formalisierung über Hemi-Metriken und Homotopie-Pfade
%	\end{itemize}
%\end{frame}

\begingroup
\frametitle{Zwei Perspektiven auf Homotopie}
\begin{frame}
	\centering
	\begin{tikzpicture}[node distance=0.8cm and 8cm, on grid]
		
		% Intrinsic box (immer sichtbar ab Folie 1)
		\uncover<1->{
			\node[draw=blue!70!black, fill=blue!10, rounded corners=3pt, thick, 
			minimum width=4cm, minimum height=2.8cm, align=center] (intrinsic) {
				{\scriptsize \textbf{Intrinsic Homotopy}} \\[0.2em]
				{\tiny Representational Similarity} \\[0.2em]
				{\tiny Ziel: Vergleich \emph{ohne} Downstream-Task} \\[0.3em]
				{\tiny  Lerne Transformation $\psi \in \text{Aff}$ mit $\psi(g(x)) \approx h(x)$}\\[0.2em]
				{\tiny  Deformation von Model \(g\) in \(h\)} \\[0.3em]
			};
		}
		
		% Extrinsic box (erst ab Folie 2)
		\uncover<2>{
			\node[draw=orange!80!black, fill=orange!15, rounded corners=3pt, thick, 
			minimum width=2.8cm, minimum height=2.8cm, align=center, right=of intrinsic] (extrinsic) {
				{\scriptsize \textbf{Extrinsic Homotopy}} \\[0.2em]
				{\tiny Functional Similarity} \\[0.2em]
				{\tiny Ziel: Vergleich auf Downstream-Task} \\[0.3em]
				{\tiny Finde $\psi, \varphi \in \mathcal{V}_C$, sodass $\psi(h(x)) \approx \varphi(g(x))$}\\[0.3em]
				{\tiny  Deformation von Model \(g\) und \(h\)} 

			};
			
			% Pfeil erscheint ebenfalls ab Folie 2
			\draw[->, thick] (intrinsic.east) -- (extrinsic.west);
		}
		
	\end{tikzpicture}
	
	\vspace{0.6em}
	%\uncover<1->{\scriptsize Chan et al.~\cite{chan_affine_2024} unterscheiden zwischen zwei komplementären Konzepten: \textit{Representational} und \textit{Functional}.}
	\vfill
	{\tiny Quelle: Chan et al., “On Affine Homotopy between Language Encoders”, NeurIPS 2024~\cite{chan_affine_2024}}
\end{frame}
\endgroup

\begingroup
\frametitle{Ziel?}
\begin{frame}
	\begin{itemize}
		\uncover<1->{\item Ziel: Vergleich von Encoder \( h, g \in \mathcal{E}_V \) bzgl.\ ihrer \textbf{approximativen Ähnlichkeit}.}
		
		\uncover<2->{\footnotesize \item Wie gut kann \( h \) durch eine Transformation \( \psi \in \mathcal{F} \) von \( g \) approximiert werden:
		\[
		d_{\mathcal{F}}(h, g) := \inf_{\psi \in \mathcal{F}} \| h - \psi \circ g \|_\infty
		\]}
		
		\uncover<3->{\item \footnotesize Specialization-Quasiordering \( h \gtrsim_{\mathcal{F}} g \) besagt:
		\begin{itemize}
			\item \( h \) lässt sich \textbf{beliebig gut durch transformiertes \( g \)} approximieren.
			\item Jede \( \ell^\infty \)-Umgebung von \( h \) enthält ein \( \psi \circ g \)
		\end{itemize}}
		\footnotesize		
		\uncover<4->{\item Diese Relation ist eine \textbf{Preorder}:
		\begin{itemize}

			\item \textbf{Reflexiv:} \( h \gtrsim h \) %via \(\psi = \mathrm{id} \)
			\item \textbf{Transitiv:} \( h \gtrsim g,  g \gtrsim f \Longrightarrow h \gtrsim f  \) %via Dreiecksungleichung
		\end{itemize}}
	\end{itemize}
\end{frame}
\endgroup


%\begingroup
%\frametitle{Intrinsic Nonlinear Homotopy: Strukturvergleich}
%\begin{frame}
%\begin{itemize}
%	\item \textbf{Neuer Ansatz:} Vergleich durch \emph{nichtlineare}, stetige Transformationen (z.\,B. Neuronale Netze)
%	\item Fokus auf \textbf{interne Repräsentationen}(letzter Hidden Layer)

%\end{itemize}

%\begin{block}{Definition (Nichtlinear intrinsisch homotop)}
%	Zwei Encoder $h, g$ sind genau dann \emph{nichtlinear intrinsisch homotop} ($h \simeq_{\text{Lip}_1} g$), wenn gilt:
%	\[
	%\inf_{\psi \in \mathcal{F}_{\text{Lip}_1}_{\text{Lip}_1}} \| h - \psi \circ g \|_\infty = 0 \quad \text{und} \quad
%	\inf_{\phi \in \mathcal{F}_{\text{Lip}_1}} \| g - \phi \circ h \|_\infty = 0
%	\]
%\end{block}
%\end{frame}



%\begingroup
%\frametitle{Intrinsic Nonlinear Homotopy: Strukturvergleich (nichtlinear)}
%\begin{frame}
%	\begin{itemize}
%		\item \textbf{Neuer Ansatz:} Vergleich durch \emph{nichtlineare}, stetige 1-Lipschitz-Transformationen (z.\,B. Neuronale Netze)
%		\item Fokus auf \textbf{interne Repräsentationen} $h(x) \in \mathbb{R}^d$ (letzter Hidden Layer)
%		\item \textbf{Intuition:} Encoder $h$ lässt sich durch eine stetige Lipschitz-Transformation aus $g$ approximieren (und umgekehrt)
%	\end{itemize}
	
%	\begin{block}{Definition (Nichtlineare intrinsische Relation)}
%		Ein Encoder $h$ ist \emph{nichtlinear intrinsisch verwandt} zu $g$ (\( h \gtrsim_{\text{Lip}_1} g \)), wenn gilt:
%		\[
%		\inf_{\psi \in \mathcal{F}_{\text{Lip}_1}} \| h - \psi \circ g \|_\infty = 0
%		\]
%	\end{block}
	
%	\begin{block}{Definition (Nichtlinear intrinsisch homotop)}
%		Zwei Encoder $h, g$ sind genau dann \emph{nichtlinear intrinsisch homotop} (\( h \simeq_{\text{Lip}_1} g \)), wenn gilt:
%		\[
%		\inf_{\psi \in \mathcal{F}_{\text{Lip}_1}} \| h - \psi \circ g \|_\infty = 0 \quad \text{und} \quad
%		\inf_{\phi \in \mathcal{F}_{\text{Lip}_1}} \| g - \phi \circ h \|_\infty = 0
%		\]
%	\end{block}
%\end{frame}
%\endgroup

\begin{frame}
\begin{itemize}
\uncover<1->{\item Nach Lemma: Wenn $(\mathcal{E}_V, d_\mathcal{F})$ ein Hemi-Metrikraum ist, dann induziert die Specialization-Quasiordering eine \textbf{Preorder}. Dann gilt für alle $g,h\in \mathcal{E}_V}$
\[
h \gtrsim_{\mathcal{E}_V} g \Longleftrightarrow d_\mathcal{F}(h,g) = 0
\]
\uncover<2->{\item Konstruktion einer Preorder auf dem Encoder-Raum induziert durch: 
	\[
	d_\mathcal{F}(h,g):= \inf_{\psi \in \mathcal{F}_{\text{Lip}_1}} \| h - \psi \circ g \|_\infty
		\]}

%\uncover<3->{	\item \( h \gtrsim_{\text{Lip}_1} g \): \emph{Jede offene Umgebung von \( h \) enthält auch \( g \)}, bezogen auf die von \( d \) induzierte Topologie.}

%		\uncover<4->{\item Um dies zu rechtfertigen, zeigen wir, dass \( d \) eine \textbf{Hemimetrik} ist.  \( \Longrightarrow h \gtrsim_{\text{Lip}_1} g \iff d_\mathcal{F}(h,g) = 0 \) ist Preorder.}

\end{itemize}

\end{frame}

\begingroup
\frametitle{Intrinsic Nonlinear Homotopy}
\begin{frame}
	\begin{block}{Intrinsic Preorder}
		Sei \( \mathcal{F}_{\text{Lip}_1} \subset \mathcal{C}(V, V) \) die Menge aller 1-Lipschitz stetigen Transformationen von \( V \) auf sich selbst\footnote{\tiny Im Beweis sehen wir, warum die Einschränkung auf 1-Lipschitz nötig ist und nicht alle stetigen Abbildungen zulässig sind.}.
		
		Die Relation \( \gtrsim_{\text{Lip}_1} \) definiert durch:
		\[
		h \gtrsim_{\text{Lip}_1} g \quad \Leftrightarrow \quad \inf_{\psi \in \mathcal{F}_{\text{Lip}_1}} \| h - \psi \circ g \|_\infty = 0
		\]
		ist eine Preorder auf \( \mathcal{E}_V \).
	\end{block}
\end{frame}
\endgroup

\begin{frame}
	\begin{itemize}
		\uncover<1->{\item Nach Lemma: Wenn $(\mathcal{E}_V, d_\mathcal{F})$ ein Hemi-Metrikraum ist, dann induziert die Specialization-Quasiordering eine \textbf{Preorder}. Dann gilt für alle $g,h\in \mathcal{E}_V}$
		\[
		h \gtrsim_{\mathcal{E}_V} g \Longleftrightarrow d_\mathcal{F}(h,g) = 0
		\]
		
		%\uncover<3->{	\item \( h \gtrsim_{\text{Lip}_1} g \): \emph{Jede offene Umgebung von \( h \) enthält auch \( g \)}, bezogen auf die von \( d \) induzierte Topologie.}
		
		%		\uncover<4->{\item Um dies zu rechtfertigen, zeigen wir, dass \( d \) eine \textbf{Hemimetrik} ist.  \( \Longrightarrow h \gtrsim_{\text{Lip}_1} g \iff d_\mathcal{F}(h,g) = 0 \) ist Preorder.}
		
	\end{itemize}
	
\end{frame}
	




\begingroup
\frametitle{\( d_{\mathcal{F}} \) ist Hemi-Metrik}

\begin{frame}
	\tiny{Hemi-Metrik-Eigenschaft (H1)}
		\textbf{(Nullabstand zur Selbstabbildung)}  
		\[
		d_{{\mathcal{F}_{\text{Lip}_1}}}(h, h) = \inf_{\psi \in \mathcal{F}} \| h - \psi \circ h \|_\infty = 0
		\]
	
	Wähle als Kandidat \(\psi = \mathrm{id} \in \mathcal{F}_{\text{Lip}_1}\). Dann gilt:
		\[
		\| h - \psi(h) \|_\infty = \| h - h \|_\infty = 0.
		\]
	
	\vspace{1em}
	\textit{Damit ist die erste Bedingung erfüllt.}
\end{frame}
\endgroup



\begingroup
\frametitle{Beweis: \( d_{{\mathcal{F}_{\text{Lip}_1}}} \) ist Hemi-Metrik}
\begin{frame}
	\tiny
	\uncover<1->{
		\textbf{Hemi-Metrik-Eigenschaft (H2)} \quad (Dreiecksungleichung)  
	}
	
	\uncover<2->{
		Für Encoder \( h, g, f \in \mathcal{E}_V \) und \( \varepsilon > 0 \) gilt:
		\vspace{-0.5em}
		\[
		d_{{\mathcal{F}_{\text{Lip}_1}}}(h, f) \leq d_{{\mathcal{F}_{\text{Lip}_1}}}(h, g) + d_{\mathcal{F}_{\text{Lip}_1}}(g, f)
		\]
	}
	
	\uncover<3->{
		\vspace{-1em}
		Wähle \( \psi_1, \psi_2 \in \mathcal{F}_{\text{Lip}_1} \) mit:
		\[
		\| h - \psi_1(g) \|_\infty \leq 	d_{{\mathcal{F}_{\text{Lip}_1}}}(h, g) + \varepsilon, \quad
		\| g - \psi_2(f) \|_\infty \leq d_{\mathcal{F}_{\text{Lip}_1}}(g, f) + \varepsilon
		\]
	}
	
	\uncover<4->{
		\vspace{-1em}
		Da \( \mathcal{F}_{\text{Lip}_1} \) unter Komposition abgeschlossen ist, betrachte:
		\[
		\psi := \psi_1 \circ \psi_2 \in \mathcal{F}_{\text{Lip}_1}
		\]
	}
	
	\uncover<5->{
		\vspace{-1em}
		Dann gilt:
		\[
		d_{{\mathcal{F}_{\text{Lip}_1}}}(h, f) \leq \| h - \psi(f) \|_\infty = \| h - \psi_1(\psi_2(f)) \|_\infty
		\]
	}
	
\end{frame}




\begin{frame}
	\tiny
	
	\uncover<1->{
		\vspace{-0.5em}
		\[
		\| h - \psi_1(\psi_2(f)) \|_\infty
		\leq 
		\underbrace{\| h - \psi_1(g) \|_\infty}_{\text{aus Approximation von } d(h, g)}
		+
		\underbrace{\| \psi_1(g) - \psi_1(\psi_2(f)) \|_\infty}_{\leq \| g - \psi_2(f) \|_\infty \text{ (1-Lipschitz von } \psi_1 \text{*})}
		\]
	}
	
	\uncover<2->{
		\vspace{-1em}
		\[
		\leq \| h - \psi_1(g) \|_\infty + \| g - \psi_2(f) \|_\infty
		\]
	}
	
	\uncover<3->{
		\vspace{-1em}
		\[
		\Rightarrow d_{\mathcal{F}_{\text{Lip}_1}}(h, f)
		\leq d_{\mathcal{F}_{\text{Lip}_1}}(h, g) + d_{\mathcal{F}_{\text{Lip}_1}}(g, f)  + 2\varepsilon 
		\]
	}
	
	\uncover<4->{
		\vspace{-1em}
		Da \( \varepsilon > 0 \) beliebig, folgt die Dreiecksungleichung.
	}
	
	\uncover<1->{
		\vspace{-0.5em}
		\begin{itemize}
			\tiny
			\item[*] \( \psi_1 \in \mathcal{F}_{\text{Lip}_1} \Rightarrow \) 1-Lipschitz:  
			\( \| \psi_1(g) - \psi_1(\psi_2(f)) \|_\infty \leq \| g - \psi_2(f) \|_\infty \)
		\end{itemize}
	}
	
	\uncover<6->{
		\vspace{-0.5em}
		\begin{itemize}
			\tiny
			\item[$\Rightarrow$] \( (\mathcal{E}_V, d_{\mathcal{F}_{\text{Lip}_1}}) \) ist Hemi-Metrischer Raum:
			\begin{itemize}
				\tiny
				\item Reflexiv: \( h \gtrsim h \) via \( \psi = \mathrm{id} \)
				\item Transitiv: \( h \gtrsim g, g \gtrsim f \Rightarrow h \gtrsim f \)
			\end{itemize}
		\end{itemize}
	}
	
\end{frame}

\endgroup

\begingroup
\frametitle{Intrinsic Nonlinear Homotopy}
\begin{frame}
	\uncover<1->{	\begin{itemize}
			\item \textbf{Ziel:} Vergleich \emph{ohne} Downstream-Task
			\item Fokus auf \textbf{interne Repräsentationen} (letzter Hidden Layer)

	\end{itemize}}
	
	\uncover<2->{	\begin{block}{Definition (Exakte Intrinsische Nicht-lineare Homotopy)}
			Zwei Encoder $h, g$ sind genau dann \emph{exakt intrinsisch nicht-linear homotop} ($h \gtrsim_{\text{Lip}_1} g$), wenn gilt:
			\[
			\inf_{\psi \in \mathrm{\text{Lip}_1}(V)} \| h - \psi \circ g \|_\infty = 0 \quad \text{und} \quad
			\inf_{\varphi \in \mathrm{\text{Lip}_1}(V)} \| g - \varphi \circ h \|_\infty = 0
			\]
	\end{block}}
\end{frame}

\endgroup

